% Dans l'introduction, on présente le problème étudié et les buts
% poursuivis. L'introduction permet de faire connaître le cadre de la
% recherche et d'en préciser le domaine d'application. Elle fournit
% les précisions nécessaires en ce qui concerne le contexte de
% réalisation de la recherche, l'approche envisagée, l'évolution de
% la réalisation. En fait, l'introduction présente au lecteur ce
% qu'il doit savoir pour comprendre la recherche et en connaître la
% portée.
\Chapter{INTRODUCTION}\label{sec:Introduction}  % 10-12 lignes pour introduire le sujet.

%%
%%  Contexte et problématique
%%

\section{Contexte et problématique}

La valeur du marché des solutions \texttt{ERP} s'établissait autour de 40 milliards \texttt{USD} mondialement en 2020 \cite{mordorintelligence_erp_2023,bigbang_erp_2023}. Le coût moyen par utilisateur, sur 5 ans, s'élevait à 9 000\$, pour une \texttt{PME}, en 2022 \cite{softwarepath_erp_2023}.

Le développement de système \texttt{ERP} est complexe, nécessite une maintenance exigeante et le risque d’introduire des erreurs est important. Comment accélérer le développement de fonctionnalités de la plateforme \texttt{ERP} Odoo\footnote{Anciennement OpenERP, ERP libre web, lien du projet \url{https://github.com/OCA/OCB}} 12 communautaire?

La plateforme ERPLibre\footnote{Projet automatisé v1.5.0 contenant Odoo 12.0 \url{https://erplibre.ca}} a été créée dans l’objectif d’accélérer le développement de la plateforme Odoo 12 communautaire. Ce mémoire va mettre l’accent sur la génération de code par des techniques de rétro-ingénierie et la gestion d’une communauté dans un contexte d’un projet logiciel libre.

% TODO problématique de devoir adapter les processus de l'entreprise aux capacité du ERP au lieu de donner un processus personnalisé

% TODO point de départ : «Code projet initial générateur de code» et donner REF

% Projet morceau de l'automate, son fondement libre, par sa création de technopoïèse.
% Il utiliser pour la liberté de l'automate d’offrir librement accès à ses fonctionnalités
% Il étudier comprendre et accepter son fonctionnement
% Il copier pour se l’approprier en respectant le copyright
% Il modifier son amélioration en tant que développement logiciel

% Combien ça prend de temps pour développer la migration pour autant de fonctionnalités? Ajouter des champs, migrer des champs en changeant leur nature.

\subsection{Choix de la plateforme \texttt{ERP}}

Choisir une plateforme \texttt{ERP} libre peut offrir des avantages significatifs en termes de coût, de flexibilité, de sécurité, de communauté et d'indépendance. Odoo a été choisi puisqu'il répondait à ces critères, cependant quelle est la version qui offre le plus de fonctionnalité?

\begin{table}
\caption{Tableau des dates de lancement du logiciel Odoo}
\centering
\begin{tabular}{|c|c|l|l|}

\hline
Légende & \cellcolor[HTML]{d9ead3}{\shortstack[l]{Version actuelle}} & \cellcolor[HTML]{fff2cc}{\shortstack[l]{Anciennes versions avec \\ maintenance étendu}} & \cellcolor[HTML]{f4cccc}{\shortstack[l]{Anciennes versions ou \\ fin de maintenance}}\\\hline

\hline\rowcolor[gray]{0.8}\color{black}
Odoo version & Date de lancement & \multicolumn{2}{|l|}{Commentaire}\\\hline
\cellcolor[HTML]{f4cccc}6.0/6.1 & octobre 2009 & \multicolumn{2}{|l|}{Première publication sous \texttt{AGPL}, premier client web}\\\hline
\cellcolor[HTML]{f4cccc}7.0 & décembre 2012 & \multicolumn{2}{|l|}{} \\\hline
\cellcolor[HTML]{f4cccc}8.0 & septembre 2014 & \multicolumn{2}{|l|}{\shortstack[l]{Changement de nom pour Odoo, anciennement \\ OpenERP}}\\\hline
\cellcolor[HTML]{f4cccc}9.0 & novembre 2015 & \multicolumn{2}{|l|}{\shortstack[l]{Première publication des éditions «Community» sous \\ licence \texttt{LGPLV3} et «Enterprise» sous licence \\ propriétaire.}}\\\hline
\cellcolor[HTML]{f4cccc}10.0 & octobre 2016 & \multicolumn{2}{|l|}{} \\\hline
\cellcolor[HTML]{f4cccc}11.0 & octobre 2017 & \multicolumn{2}{|l|}{} \\\hline
\cellcolor[HTML]{f4cccc}12.0 & octobre 2018 & \multicolumn{2}{|l|}{Version utilisé dans ERPLibre 1.5.0}\\\hline
\cellcolor[HTML]{fff2cc}13.0 & octobre 2019 & \multicolumn{2}{|l|}{}\\\hline
\cellcolor[HTML]{fff2cc}14.0 & octobre 2020 & \multicolumn{2}{|l|}{} \\\hline
\cellcolor[HTML]{fff2cc}15.0 & octobre 2021 & \multicolumn{2}{|l|}{} \\\hline
\cellcolor[HTML]{d9ead3}16.0 & octobre 2022 & \multicolumn{2}{|l|}{} \\\hline
\end{tabular}
\label{tab:choix_plateform_erp}
\end{table}

% Regarder l’évolution des modules dans \texttt{OCA}, en prenant leur nom de module, pour chaque version d’Odoo, selon une date déterminée, regarder le nombre de fois qu’il se répète. Ceci va indiquer la vitesse de migration des modules dans la communauté.

En janvier 2023, les versions 9.0 à 12.0 ne sont plus supportées officiellement par la compagnie Odoo, voir tableau~\ref{tab:choix_plateform_erp}, mais elles le sont encore par \texttt{OCA}. La version 16.0 est la version stable actuelle. La démonstration commence à partir de 9.0, là où débute la divergence entre une version communautaire et entreprise.

Au printemps 2020, Odoo version 12.0 a été choisi par ERPLibre\footnote{Première version de ERPLibre : \url{https://github.com/ERPLibre/ERPLibre/releases/tag/v0.1.0}.}. Une recherche de module par version d'Odoo a été effectué sur 11 Go de code et de données sur le projet ERPLibre version 1.5.0, voir le tableau~\ref{tab:nb_module_version_odoo}. Ainsi, en date du 1 janvier 2023, la version 12.0 est encore le bon choix avec 2 977 modules, puisqu'il est celui qu'il a le plus de module sur les 133 répertoires gérés par ERPLibre. Cette tendance pourrait changer en 2024 selon l’évolution.

Pour obtenir les résultats du tableau~\ref{tab:nb_module_version_odoo}, un script a été développé pour chercher la quantité de modules en cherchant dans les 133 répertoires Git\footnote{Logiciel de gestion de versions décentralisé}, puis pour toutes les versions d'Odoo, pour tous les modules qui contiennent le fichier «manifest» et que ceux-ci inclus le paramètre qu'il est installable, à date précédente du 1 janvier de chaque année.
% le fichier «\_\_openerp\_\_.py» ou «\_\_manifest\_\_.py»

Des fois, la quantité de module diminue d'une année à l'autre. Il y a création d'une nouvelle branche lors d'une nouvelle version qui est la suite de la version précédente. Par exemple, dans le tableau~\ref{tab:nb_module_version_odoo}, la version 10.0 entre 2017 et 2018, il y a une réduction de 171 modules dans les répertoires d'entreprise, mais il y a eu seulement 4 mois pour faire le nettoyage, les méthodes de mises à jours ont évolués depuis.

De plus, les chiffres du tableau~\ref{tab:nb_module_version_odoo} semblent démontrer que les versions paires d'Odoo sont plus populaire que les versions impaires. Cependant la communauté d'Odoo est bien plus grosse que 133 répertoires.

Dans la section total du tableau~\ref{tab:nb_module_version_odoo}, la section unique signifie que la somme va ignorer les doublons. En date du 1 janvier 2023, il y a au total 17 309 modules, mais 6 063 modules uniques. Ça signifie qu'il y a 11 246 modules en doublon. Hors, le code diffère d'une version à l'autre même si c'est un doublon, ils peuvent avoir des bogues ou des fonctionnalités différentes entre eux.

% Prochain tableau
\begin{table}
\caption{Nombre de modules par version Odoo à partir du 1 janvier minuit par année sur la plateforme ERPLibre 1.5.0.}
\centering
\begin{tabular}{|l|l|l|l|l|l|l|l|l|}
\hline

Légende & \multicolumn{2}{|l|}{\shortstack[l]{Total\\\textcolor[HTML]{274e13}{OCA}\\\textcolor[HTML]{4c1130}{Entreprise}}} & \multicolumn{2}{|c|}{\cellcolor[HTML]{d9ead3}{\shortstack[l]{Version \\ actuelle}}} & \multicolumn{2}{|c|}{\cellcolor[HTML]{fff2cc}{\shortstack[l]{Anciennes \\ versions avec \\ maintenance \\ étendu}}} & \multicolumn{2}{|c|}{\cellcolor[HTML]{f4cccc}{\shortstack[l]{Anciennes \\ versions ou \\ fin de \\ maintenance}}}\\\hline

\multicolumn{9}{|l|}{\shortstack[l]{17309/\textcolor[HTML]{274e13}{10728}/\textcolor[HTML]{4c1130}{6581} modules à supporter le 1 janvier 2023 \\ 17465/\textcolor[HTML]{274e13}{10952}/\textcolor[HTML]{4c1130}{6513} modules le 15 février 2023 \\ 156/\textcolor[HTML]{274e13}{132}/\textcolor[HTML]{4c1130}{24} nouveaux modules en 31 jours durant janvier 2023}}\\\hline

Odoo version & 2016 & 2017 & 2018 & 2019 & 2020 & 2021 & 2022 & 2023 \\\hline

6.1 &
\cellcolor[HTML]{fff2cc}{\shortstack[r]{295 \\ \textcolor[HTML]{274e13}{269} \\ \textcolor[HTML]{4c1130}{26}}} & 
\cellcolor[HTML]{f4cccc}{\shortstack[r]{299 \\ \textcolor[HTML]{274e13}{270} \\ \textcolor[HTML]{4c1130}{29}}} &
\cellcolor[HTML]{f4cccc}{\shortstack[r]{299 \\ \textcolor[HTML]{274e13}{270} \\ \textcolor[HTML]{4c1130}{29}}} &
\cellcolor[HTML]{f4cccc}{\shortstack[r]{299 \\ \textcolor[HTML]{274e13}{270} \\ \textcolor[HTML]{4c1130}{36}}} &
\cellcolor[HTML]{f4cccc}{\shortstack[r]{299 \\ \textcolor[HTML]{274e13}{270} \\ \textcolor[HTML]{4c1130}{36}}} &
\cellcolor[HTML]{f4cccc}{\shortstack[r]{299 \\ \textcolor[HTML]{274e13}{270} \\ \textcolor[HTML]{4c1130}{36}}} &
\cellcolor[HTML]{f4cccc}{\shortstack[r]{299 \\ \textcolor[HTML]{274e13}{270} \\ \textcolor[HTML]{4c1130}{36}}} &
\cellcolor[HTML]{f4cccc}{\shortstack[r]{299 \\ \textcolor[HTML]{274e13}{270} \\ \textcolor[HTML]{4c1130}{36}}} \\\hline

7.0 &
\cellcolor[HTML]{fff2cc}{\shortstack[r]{637 \\ \textcolor[HTML]{274e13}{597} \\ \textcolor[HTML]{4c1130}{40}}} & 
\cellcolor[HTML]{fff2cc}{\shortstack[r]{633 \\ \textcolor[HTML]{274e13}{592} \\ \textcolor[HTML]{4c1130}{41}}} &
\cellcolor[HTML]{f4cccc}{\shortstack[r]{634 \\ \textcolor[HTML]{274e13}{593} \\ \textcolor[HTML]{4c1130}{41}}} &
\cellcolor[HTML]{f4cccc}{\shortstack[r]{635 \\ \textcolor[HTML]{274e13}{594} \\ \textcolor[HTML]{4c1130}{41}}} &
\cellcolor[HTML]{f4cccc}{\shortstack[r]{669 \\ \textcolor[HTML]{274e13}{619} \\ \textcolor[HTML]{4c1130}{50}}} &
\cellcolor[HTML]{f4cccc}{\shortstack[r]{669 \\ \textcolor[HTML]{274e13}{619} \\ \textcolor[HTML]{4c1130}{50}}} &
\cellcolor[HTML]{f4cccc}{\shortstack[r]{669 \\ \textcolor[HTML]{274e13}{619} \\ \textcolor[HTML]{4c1130}{50}}} &
\cellcolor[HTML]{f4cccc}{\shortstack[r]{671 \\ \textcolor[HTML]{274e13}{619} \\ \textcolor[HTML]{4c1130}{52}}} \\\hline

8.0 &
\cellcolor[HTML]{fff2cc}{\shortstack[r]{741 \\ \textcolor[HTML]{274e13}{597} \\ \textcolor[HTML]{4c1130}{144}}} & 
\cellcolor[HTML]{fff2cc}{\shortstack[r]{1092 \\ \textcolor[HTML]{274e13}{907} \\ \textcolor[HTML]{4c1130}{185}}} &
\cellcolor[HTML]{fff2cc}{\shortstack[r]{1215 \\ \textcolor[HTML]{274e13}{996} \\ \textcolor[HTML]{4c1130}{219}}} &
\cellcolor[HTML]{f4cccc}{\shortstack[r]{1265 \\ \textcolor[HTML]{274e13}{1027} \\ \textcolor[HTML]{4c1130}{238}}} &
\cellcolor[HTML]{f4cccc}{\shortstack[r]{1290 \\ \textcolor[HTML]{274e13}{1036} \\ \textcolor[HTML]{4c1130}{254}}} &
\cellcolor[HTML]{f4cccc}{\shortstack[r]{1297 \\ \textcolor[HTML]{274e13}{1043} \\ \textcolor[HTML]{4c1130}{254}}} &
\cellcolor[HTML]{f4cccc}{\shortstack[r]{1340 \\ \textcolor[HTML]{274e13}{1049} \\ \textcolor[HTML]{4c1130}{291}}} &
\cellcolor[HTML]{f4cccc}{\shortstack[r]{1341 \\ \textcolor[HTML]{274e13}{1050} \\ \textcolor[HTML]{4c1130}{291}}} \\\hline

9.0 &
\cellcolor[HTML]{d9ead3}{\shortstack[r]{135 \\ \textcolor[HTML]{274e13}{46} \\ \textcolor[HTML]{4c1130}{89}}} & 
\cellcolor[HTML]{fff2cc}{\shortstack[r]{456 \\ \textcolor[HTML]{274e13}{346} \\ \textcolor[HTML]{4c1130}{110}}} &
\cellcolor[HTML]{fff2cc}{\shortstack[r]{725 \\ \textcolor[HTML]{274e13}{602} \\ \textcolor[HTML]{4c1130}{123}}} &
\cellcolor[HTML]{fff2cc}{\shortstack[r]{776 \\ \textcolor[HTML]{274e13}{643} \\ \textcolor[HTML]{4c1130}{133}}} &
\cellcolor[HTML]{f4cccc}{\shortstack[r]{796 \\ \textcolor[HTML]{274e13}{659} \\ \textcolor[HTML]{4c1130}{137}}} &
\cellcolor[HTML]{f4cccc}{\shortstack[r]{803 \\ \textcolor[HTML]{274e13}{666} \\ \textcolor[HTML]{4c1130}{137}}} &
\cellcolor[HTML]{f4cccc}{\shortstack[r]{844 \\ \textcolor[HTML]{274e13}{666} \\ \textcolor[HTML]{4c1130}{178}}} &
\cellcolor[HTML]{f4cccc}{\shortstack[r]{850 \\ \textcolor[HTML]{274e13}{669} \\ \textcolor[HTML]{4c1130}{181}}} \\\hline

10.0 &
\cellcolor[HTML]{cccccc}{} & 
\cellcolor[HTML]{d9ead3}{\shortstack[r]{523 \\ \textcolor[HTML]{274e13}{111} \\ \textcolor[HTML]{4c1130}{412}}} &
\cellcolor[HTML]{fff2cc}{\shortstack[r]{954 \\ \textcolor[HTML]{274e13}{713} \\ \textcolor[HTML]{4c1130}{241}}} &
\cellcolor[HTML]{fff2cc}{\shortstack[r]{1 537 \\ \textcolor[HTML]{274e13}{953} \\ \textcolor[HTML]{4c1130}{584}}} &
\cellcolor[HTML]{fff2cc}{\shortstack[r]{1 647 \\ \textcolor[HTML]{274e13}{1 047} \\ \textcolor[HTML]{4c1130}{600}}} &
\cellcolor[HTML]{f4cccc}{\shortstack[r]{1 685 \\ \textcolor[HTML]{274e13}{1 085} \\ \textcolor[HTML]{4c1130}{600}}} &
\cellcolor[HTML]{f4cccc}{\shortstack[r]{1 754 \\ \textcolor[HTML]{274e13}{1 109} \\ \textcolor[HTML]{4c1130}{645}}} &
\cellcolor[HTML]{f4cccc}{\shortstack[r]{1 765 \\ \textcolor[HTML]{274e13}{1 120} \\ \textcolor[HTML]{4c1130}{645}}} \\\hline

11.0 &
\cellcolor[HTML]{cccccc}{} & 
\cellcolor[HTML]{cccccc}{} &
\cellcolor[HTML]{d9ead3}{\shortstack[r]{288 \\ \textcolor[HTML]{274e13}{77} \\ \textcolor[HTML]{4c1130}{211}}} &
\cellcolor[HTML]{fff2cc}{\shortstack[r]{1 398 \\ \textcolor[HTML]{274e13}{658} \\ \textcolor[HTML]{4c1130}{740}}} &
\cellcolor[HTML]{fff2cc}{\shortstack[r]{1 710 \\ \textcolor[HTML]{274e13}{929} \\ \textcolor[HTML]{4c1130}{781}}} &
\cellcolor[HTML]{fff2cc}{\shortstack[r]{1 797 \\ \textcolor[HTML]{274e13}{1 000} \\ \textcolor[HTML]{4c1130}{797}}} &
\cellcolor[HTML]{f4cccc}{\shortstack[r]{1 860 \\ \textcolor[HTML]{274e13}{1 023} \\ \textcolor[HTML]{4c1130}{837}}} &
\cellcolor[HTML]{f4cccc}{\shortstack[r]{1 869 \\ \textcolor[HTML]{274e13}{1 032} \\ \textcolor[HTML]{4c1130}{864}}} \\

\noalign{\hrule height 2pt}
\multicolumn{1}{!{\vrule width 2pt}l!{\vrule width 1pt}}{\textbf{12.0}} &
\cellcolor[HTML]{cccccc}{} & 
\cellcolor[HTML]{cccccc}{} &
\cellcolor[HTML]{cccccc}{} &
\multicolumn{1}{!{\vrule width 2pt}r!{\vrule width 1pt}}{\textbf{\cellcolor[HTML]{d9ead3}{\shortstack[r]{784 \\ \textcolor[HTML]{274e13}{137} \\ \textcolor[HTML]{4c1130}{647}}}}} &
\multicolumn{1}{!{\vrule width 2pt}r!{\vrule width 1pt}}{\textbf{\cellcolor[HTML]{fff2cc}{\shortstack[r]{1 837 \\ \textcolor[HTML]{274e13}{993} \\ \textcolor[HTML]{4c1130}{844}}}}} &
\multicolumn{1}{!{\vrule width 2pt}r!{\vrule width 1pt}}{\textbf{\cellcolor[HTML]{fff2cc}{\shortstack[r]{2 503 \\ \textcolor[HTML]{274e13}{1 464} \\ \textcolor[HTML]{4c1130}{1 039}}}}} &
\multicolumn{1}{!{\vrule width 2pt}r!{\vrule width 1pt}}{\textbf{\cellcolor[HTML]{fff2cc}{\shortstack[r]{2 851 \\ \textcolor[HTML]{274e13}{1 633} \\ \textcolor[HTML]{4c1130}{1 218}}}}} &
\multicolumn{1}{!{\vrule width 2pt}r!{\vrule width 1pt}}{\textbf{\cellcolor[HTML]{f4cccc}{\shortstack[r]{2 977 \\ \textcolor[HTML]{274e13}{1 693} \\ \textcolor[HTML]{4c1130}{1 284}}}}} \\
\noalign{\hrule height 2pt}

13.0 &
\cellcolor[HTML]{cccccc}{} & 
\cellcolor[HTML]{cccccc}{} &
\cellcolor[HTML]{cccccc}{} &
\cellcolor[HTML]{cccccc}{} &
\cellcolor[HTML]{d9ead3}{\shortstack[r]{617 \\ \textcolor[HTML]{274e13}{115} \\ \textcolor[HTML]{4c1130}{502}}} &
\cellcolor[HTML]{fff2cc}{\shortstack[r]{1 445 \\ \textcolor[HTML]{274e13}{844} \\ \textcolor[HTML]{4c1130}{601}}} &
\cellcolor[HTML]{fff2cc}{\shortstack[r]{2 024 \\ \textcolor[HTML]{274e13}{1 310} \\ \textcolor[HTML]{4c1130}{714}}} &
\cellcolor[HTML]{fff2cc}{\shortstack[r]{2 241 \\ \textcolor[HTML]{274e13}{1 506} \\ \textcolor[HTML]{4c1130}{735}}} \\\hline

14.0 &
\cellcolor[HTML]{cccccc}{} & 
\cellcolor[HTML]{cccccc}{} &
\cellcolor[HTML]{cccccc}{} &
\cellcolor[HTML]{cccccc}{} &
\cellcolor[HTML]{cccccc}{} &
\cellcolor[HTML]{d9ead3}{\shortstack[r]{906 \\ \textcolor[HTML]{274e13}{129} \\ \textcolor[HTML]{4c1130}{777}}} &
\cellcolor[HTML]{fff2cc}{\shortstack[r]{2 150 \\ \textcolor[HTML]{274e13}{1 143} \\ \textcolor[HTML]{4c1130}{1 007}}} &
\cellcolor[HTML]{fff2cc}{\shortstack[r]{2 648 \\ \textcolor[HTML]{274e13}{1 698} \\ \textcolor[HTML]{4c1130}{950}}} \\\hline

15.0 &
\cellcolor[HTML]{cccccc}{} & 
\cellcolor[HTML]{cccccc}{} &
\cellcolor[HTML]{cccccc}{} &
\cellcolor[HTML]{cccccc}{} &
\cellcolor[HTML]{cccccc}{} &
\cellcolor[HTML]{cccccc}{} &
\cellcolor[HTML]{d9ead3}{\shortstack[r]{786 \\ \textcolor[HTML]{274e13}{96} \\ \textcolor[HTML]{4c1130}{690}}} &
\cellcolor[HTML]{fff2cc}{\shortstack[r]{1 669 \\ \textcolor[HTML]{274e13}{865} \\ \textcolor[HTML]{4c1130}{804}}} \\\hline

16.0 &
\cellcolor[HTML]{cccccc}{} & 
\cellcolor[HTML]{cccccc}{} &
\cellcolor[HTML]{cccccc}{} &
\cellcolor[HTML]{cccccc}{} &
\cellcolor[HTML]{cccccc}{} &
\cellcolor[HTML]{cccccc}{} &
\cellcolor[HTML]{cccccc}{} &
\cellcolor[HTML]{d9ead3}{\shortstack[l]{972 \\ \textcolor[HTML]{274e13}{206} \\ \textcolor[HTML]{4c1130}{766}}} \\\hline

\multicolumn{9}{|c|}{Total}\\\hline

Somme &
\shortstack[r]{1 808 \\ \textcolor[HTML]{274e13}{1 509} \\ \textcolor[HTML]{4c1130}{299}} & 
\shortstack[r]{3 003 \\ \textcolor[HTML]{274e13}{2 226} \\ \textcolor[HTML]{4c1130}{777}} &
\shortstack[r]{4 115 \\ \textcolor[HTML]{274e13}{3 251} \\ \textcolor[HTML]{4c1130}{864}} &
\shortstack[r]{6 701 \\ \textcolor[HTML]{274e13}{4 282} \\ \textcolor[HTML]{4c1130}{2 419}} &
\shortstack[r]{8 872 \\ \textcolor[HTML]{274e13}{5 668} \\ \textcolor[HTML]{4c1130}{3 204}} &
\shortstack[r]{11 411 \\ \textcolor[HTML]{274e13}{7 120} \\ \textcolor[HTML]{4c1130}{4 291}} &
\shortstack[r]{14 584 \\ \textcolor[HTML]{274e13}{8 918} \\ \textcolor[HTML]{4c1130}{5 666}} &
\shortstack[r]{17 309 \\ \textcolor[HTML]{274e13}{10 728} \\ \textcolor[HTML]{4c1130}{6 581}} \\\hline

Support Odoo &
\shortstack[r]{1 808 \\ \textcolor[HTML]{274e13}{1 509} \\ \textcolor[HTML]{4c1130}{299}} & 
\shortstack[r]{2 704 \\ \textcolor[HTML]{274e13}{1 956} \\ \textcolor[HTML]{4c1130}{748}} &
\shortstack[r]{3 182 \\ \textcolor[HTML]{274e13}{2 388} \\ \textcolor[HTML]{4c1130}{794}} &
\shortstack[r]{4 495 \\ \textcolor[HTML]{274e13}{2 391} \\ \textcolor[HTML]{4c1130}{2 104}} &
\shortstack[r]{5 811 \\ \textcolor[HTML]{274e13}{3 084} \\ \textcolor[HTML]{4c1130}{2 727}} &
\shortstack[r]{6 651 \\ \textcolor[HTML]{274e13}{3 437} \\ \textcolor[HTML]{4c1130}{3 214}} &
\shortstack[r]{7 811 \\ \textcolor[HTML]{274e13}{4 182} \\ \textcolor[HTML]{4c1130}{3 629}} &
\shortstack[r]{7 530 \\ \textcolor[HTML]{274e13}{4 275} \\ \textcolor[HTML]{4c1130}{3 255}} \\\hline

Unique &
\shortstack[r]{1 244 \\ \textcolor[HTML]{274e13}{1 047} \\ \textcolor[HTML]{4c1130}{197}} & 
\shortstack[r]{1 995 \\ \textcolor[HTML]{274e13}{1 435} \\ \textcolor[HTML]{4c1130}{560}} &
\shortstack[r]{2 260 \\ \textcolor[HTML]{274e13}{1 845} \\ \textcolor[HTML]{4c1130}{415}} &
\shortstack[r]{3 214 \\ \textcolor[HTML]{274e13}{2 172} \\ \textcolor[HTML]{4c1130}{1 042}} &
\shortstack[r]{3 927 \\ \textcolor[HTML]{274e13}{2 610} \\ \textcolor[HTML]{4c1130}{1 317}} &
\shortstack[r]{4 676 \\ \textcolor[HTML]{274e13}{3 080} \\ \textcolor[HTML]{4c1130}{1 596}} &
\shortstack[r]{5 452 \\ \textcolor[HTML]{274e13}{3 572} \\ \textcolor[HTML]{4c1130}{1 880}} &
\shortstack[r]{6 063 \\ \textcolor[HTML]{274e13}{3 980} \\ \textcolor[HTML]{4c1130}{2 083}} \\\hline

\multicolumn{9}{|l|}{\shortstack[l]{133/\textcolor[HTML]{274e13}{72}/\textcolor[HTML]{4c1130}{61} répertoires de modules dans ERPLibre 1.5.0}}\\\hline

\end{tabular}
\label{tab:nb_module_version_odoo}
\end{table}

\newpage

\subsection{Introduction Accorderie}

L'Accorderie, un réseau d'entraide Québécois, était à la recherche d'une plateforme améliorée, avec des technologies plus récentes pour contrer l'effet des réseaux sociaux qui est devenu un intermédiaire intéressant pour échanger entre les membres, ainsi que d’automatiser les processus d’échange de temps, qui demande actuellement une gestion manuelle.

% L’objectif du projet en collaboration avec l’Accorderie est de faire une plateforme améliorée avec des technologies plus récentes pour contrer l’effet des réseaux sociaux qui est devenu un intermédiaire intéressant pour échanger entre les membres, ainsi que d’automatiser les processus d’échange de temps, qui demande actuellement une gestion manuelle.

% WHY samuel voulait faire une plateforme d'échange de temps et j'étais d'accord puisque c'est une manière pour promouvoir le libre auprès des communautés via des moyens technologiques et leur rendre accessible l'automate codeur pour les aider avec leur entreprise et même aider dans les mouvements en transition.

% Nous eux besoin de répondre à mettre en place un réseau d'entraide basé sur des concepts d'échange de temps. \footnote{citation de samuel à effectuer} pour répondre à des besoins qui ne peuvent pas être fait par les entreprises.
% TODO besoin de faire 

% Nous nous sommes rencontrés, avec Nadia Mohammed-Azizi, directrice du Réseau Accorderie, pour comprendre les problématiques et nous avons par la suite fait une analyse fonctionnelle.

% L’objectif du projet en collaboration avec l’Accorderie est de faire une plateforme améliorée avec des technologies plus récentes pour contrer l’effet des réseaux sociaux qui est devenu un intermédiaire intéressant pour échanger entre les membres, ainsi que d’automatiser les processus d’échange de temps, qui demande actuellement une gestion manuelle.

«Le 3 juin 2002, l’Accorderie de Québec est officiellement constituée en tant qu’organisme à but non lucratif. Sa mission : lutter contre la pauvreté et l’exclusion sociale ainsi que favoriser la mixité sociale»\cite{erudit_accorderie_2014}. 

Nous avons décelé que le présent projet pourrait répondre à leur besoin avec l'automate programmeur qui permet de facilité, entre autres la maintenance dans le temps. De plus, la plateforme a le potentiel de leur éviter des coûts, du développement redondant et leur donner des fonctionnalités personnalisées. Ainsi, nous avons obtenu accès au code source \texttt{PHP} de la plateforme Espace Membre dont le copyright mentionne l’année 2007 par la compagnie GRF Ressource Informatique. De plus, nous avons aussi eu accès à la base de données et selon les archives, le premier échange tracé est le 1 janvier 2003. La plateforme aurait eu plusieurs mises à jour au fil du temps. Cela nous donnait un cas réel empirique, avec des données et des utilisateurs réels, pour rendre concret une plateforme d'échange de temps dont le présent mémoire fait objet.

\subsection{Introduction CEPPP}

% Why projet avec SantéLibre

% how

% what

Le CEPPP, Centre d’excellence sur le partenariat avec les patients et le public, était à la recherche d'une plateforme pour la gestion du partenariat avec les patients et le public. Le Portail des partenaires (“Portail”) du Centre d’excellence sur le partenariat avec les patients et le public (CEPPP) est issu de la fusion de communautés de patients partenaires, entre autres de la Direction collaboration et partenariat patient (DCPP) de la Faculté de médecine de l’Université de Montréal, et celle du CEPPP du Centre de recherche du Centre Hospitalier de l’Université de Montréal (CR-CHUM). Le Portail est un outil qui vient soutenir les activités de recrutement et de recherche sur les pratiques de partenariat. Ils avaient donc déjà une solution\footnote{Lien du projet Git CEPPP CRM \url{https://github.com/lerenardprudent/ceppp_crm/tree/master}}, mais elle était incomplète, il manquait la gestion de l'anonymisation et l'interface ne répondait pas à des critères esthétiques.

Le présent projet venait pallier aux problèmes rencontrés en facilitant le développement, en accélérant ce dernier, en permettant une personnalisation et en développant l'anonymisation des données. À l'instar de l'Accorderie, cela permettait aussi un cas réel d'utilisation empirique du projet afin d'améliorer et de comprendre ce dont le générateur de code a besoin. 



% Selon le développement initial \footnote{Lien du projet Git \url{https://github.com/lerenardprudent/ceppp_crm/tree/master}}, le développement a été commencé le 25 mars 2019 et a terminé le 27 décembre 2019, utilisant la plateforme SuiteCRM en PHP, sous licence AGPLv3. Le codage a été fait directement sur la plateforme, rendant plus difficile la mise à jour, puisqu’il faut éviter les conflits de code.

\section{Définitions et concepts}

\subsection{Caractéristique de la plateforme Odoo}


\subsubsection{Internationalization et localisation}

Dans un réseau d'entraide, il est nécessaire de supporter plusieurs langues pour faciliter la compréhension de l'outil informatique à l'utilisation. Pour y arriver, il existe un standard \texttt{i18n}, qui a été référencé \cite{i18n_wiley}, qui permet d'adapter les logiciels informatiques à plusieurs langues sur plusieurs localisations\footnote{Wikipédia \url{https://fr.wikipedia.org/wiki/Internationalisation_et_localisation}}. Odoo rend accessible plusieurs bibliothèques pour permettre l’extraction de chaînes de caractères au moment de l’exécution, que ce soit dans du Python, Javascript ou \texttt{XML}. Le système permet de générer un fichier «pot» qui contient ces chaînes de caractères. Pour supporter une nouvelle langue et une localisation, on copie le fichier «pot» pour faire un fichier «po» sous la nomenclature \texttt{i18n} et on peut faire la traduction ou l’adaptation linguistique. Le système peut aussi générer la langue existante pour faire un fichier «po» avec les traductions déjà connu par le système, il suffit de le mettre à jour.

\subsubsection{Architecture \texttt{MVC}}

Pour générer des modules Odoo, le générateur de code a besoin de se reposer sur l'architecture \texttt{MVC} pour permettre une séparation claire des responsabilités et une structure de code facile à maintenir. Le générateur doit ainsi générer chacune des sections de cette architecture pour faire un tout qui permet d'échanger les données entre la base de données et l'interface utilisateur.

L’architecture \texttt{MVC} permet de séparer les composantes logicielles comme suit : 
\begin{enumerate}
    \item Le modèle : représente les données et les règles de l’application. Le modèle est responsable de la manipulation des données, de leur stockage et de leur récupération.
    \item La vue : représente la présentation de la donnée. La vue est responsable de l’affichage des données stockées dans le modèle à l’utilisateur final.
    \item Le contrôleur : gère les interactions entre le modèle et la vue. Le contrôleur est responsable de la réception des demandes de l’utilisateur et de leur transmission au modèle, puis de la récupération des données du modèle pour les transmettre à la vue.
\end{enumerate}
Il est possible de modifier une de ces composantes sans affecter les autres composantes, ce qui facilite sa maintenance.

\subsubsection{Website builder}

Le «Website builder» est un outil nécessaire dans un réseau d'entraide pour permettre une autonomie aux utilisateurs lambda\footnote{Utilisateur semblable à la majorité dans son comportement. \url{https://fr.wiktionary.org/wiki/utilisateur_lambda}} (augmenter l'accessibilité) dans la création et mise à jour de contenus des pages web sur le site vitrine de l'organisation. C’est un mécanisme \texttt{LCNC} dans Odoo 12 pour créer et modifier un site web multi-page par un mécanisme de «drag and drop» avec des «snippets» paramétrables. Il permet de modifier une page web en réduisant l’intervention d’un expert technique réduisant ainsi les coûts de développement.

\subsubsection{Architecture ORM}

L'utilisation de requête SQL pour communiquer avec des bases de données demandent un temps considérable au développeur à implémenter et il nécessite la programmation d'interface avec la base de données. L'utilisation d'un \texttt{ORM} permet d'augmenter la productivité, de faciliter la maintenance et augmente la sécurité d'un logiciel. L’objectif du \texttt{ORM} est de faciliter la manipulation des données stockées dans une base de données relationnelle en les représentant sous forme d’objets dans un langage de programmation orienté objet\footnote{\url{https://fr.wikipedia.org/wiki/Mapping_objet-relationnel}}. Cela permet la simplification de l’architecture en l’écrivant en langage haut niveau au lieu de directement en \texttt{SQL}, réduisant le nombre d'erreurs et facilite la maintenance.

\subsubsection{Architecture modulaire par héritage}

Le développement de système \texttt{ERP} est complexe par l'ensemble des fonctionnalités nécessaire à la gestion des procédures et ressources des organisations. Pour réduire la complexité du développement logiciel, utiliser une architecture modulaire permet la réutilisation de code et créer des relations fonctionnelles personnalisables aux contextes des organisations.

L’héritage dans Odoo 12 peut se faire de deux manières : 
\begin{enumerate}
    \item L’héritage par extension : Cela permet d’ajouter des fonctionnalités ou de modifier le comportement d’un module existant sans toucher au code source d’origine. Les nouvelles fonctionnalités peuvent être ajoutées en créant un nouveau module qui hérite du module d’origine et en y ajoutant des vues, des modèles ou des contrôleurs supplémentaires.
    \item L’héritage par substitution : Cela permet de remplacer complètement le comportement d’un module existant en créant un nouveau module qui hérite du module d’origine et en y modifiant les vues, les modèles ou les contrôleurs existants.
\end{enumerate}
En utilisant l’architecture modulaire avec héritage dans Odoo 12, les développeurs peuvent facilement personnaliser l’application pour répondre aux besoins spécifiques de leur organisation, sans toucher au code source d’origine et sans compromettre la compatibilité avec les mises à jour futures.


\subsubsection{Fonctionnalité du hook lors de l’installation d’un module}

Au moment de l’installation d’un module Odoo, il y a des opérations qui peuvent être effectué, comme par exemple migrer des données pour les adapter à un nouveau modèle de données. Ainsi, il y a une fonctionnalité qui se nomme le hook pour «pre-install», «post-install» et «uninstall». Ce sont des méthodes qui sont exécutés soit au moment de l’installation, avant ou après l’initialisation de la plateforme, puis à la désinstallation. En «post-install», il devient possible d’exécuter du code et accéder à la plupart des fonctionnalités du \texttt{ORM} au moment d’installer un module. C’est une manière pour exécuter des scripts dans la plateforme au moment de l’installation d’un module, que ce soit en ligne de commande ou via l’application «Application» dans Odoo.

\subsubsection{ERPLibre}

Depuis la version 9.0 d'Odoo, une version communautaire et entreprise ont été créés causant une divergence sur les fonctionnalités créant le modèle d'affaires «Open Core»\footnote{\url{https://fr.wikipedia.org/wiki/Open_core}}. L'adaptation d'un \texttt{ERP} est complexe et nécessite l'intervention d'un ou des experts pour bien répondre aux besoins de l'utilisateur pour la personnalisation de la plateforme au réalité de l'organisation. Bien que l'\texttt{OCA} travaille pour rendre accessible librement ces fonctionnalités, cela vient limiter les réseau d'entraides à pouvoir se débrouiller de manière souveraine.

La plateforme ERPLibre a ainsi été créé en début 2020 encapsulant la version Odoo 12.0 sous licence AGPLv3 en offrant une alternative 100\% libre, et dans le but de faciliter le déploiement et le développement pour une organisation en permettant la gestion des dépendances avec Poetry, automatisation du déploiement avec les Docker, la gestion de tous les répertoires de module~\ref{tab:nb_module_version_odoo} avec Git Repo\footnote{\url{https://gerrit.googlesource.com/git-repo}}, et plusieurs scripts pour le développement et une documentation propre pour son utilisation. Cela permet de rendre accessible à la même endroit toutes l'information nécessaire à la gestion de son ERP pour une organisation.

% \subsection{Communauté de développement}

% % TODO manque why et how
% Une communauté de développement va regrouper une communauté de développeurs (qui travaille sur l’aspect technique) et une communauté d’utilisateurs (qui travaille sur l’aspect fonctionnel). Il faut réussir à faire interagir ces deux groupes d'individus et les amener à travailler ensemble.

% \subsubsection{Générateur de code accessible dans la communauté Odoo}

% % TODO REF
% % Il permet de générer un module simple dans Odoo et faire son modèle de données et vue de base admin

\subsection{Cadre conceptuel}

Nous proposons un modèle formel de la génération de code qui guide le travail proposé dans ce mémoire. Soit C, un code informatique exécutable (qui pourrait aussi être un script interprétable), soit µ$_C$ les méta-données du code, et soit M, une machine disposant de deux modes d’opération : $M^d$ le mode direct, et $M^i$ le mode inverse, voir figure~\ref{fig:mode_direct_inverse}. Lorsque M opère en mode direct sur µ$_C$, on doit obtenir C; en opérant en mode inverse sur C, on doit obtenir µ$_C$.

On peut représenter symbolique ces deux processus par les équations
$M^d$(µC)=C et \\ $M^i$(C)=µC. La machine peut alors être représentée par M = \{$M^d$,$M^i$\}.

L’ingénierie de génération est le mode direct. La rétro-ingénierie, le mode inverse, est le processus qui consiste à examiner et à analyser un système existant pour en comprendre le fonctionnement et les spécifications. Cette boucle va permettre d’intégrer des concepts d’amélioration continue sur l’évolution du code.

\begin{figure}[htb]
\centering
\includegraphics[width=4in]{images/machine_ing_retro_ing.drawio.png}
\caption{Mode direct et inverse}
\label{fig:mode_direct_inverse}
\end{figure}

Pour concrétiser le sens de ce modèle formel, nous allons proposer quelques exemples simples. Ayant pour but de faciliter la compréhension, ces exemples sont triviaux et ne présentent pas le plein potentiel de notre approche. L’interpréteur Python 3.6 et + est utilisé pour les exemples de codes.

Pour la plupart des exemples, le C, représenté par «C.py», est le code «Hello, World!», voir Listing~\ref{lst:hello_world}.

\begin{lstlisting}[language=Python, upquote=true, caption={Exemple de code Hello, World!}, label={lst:hello_world}]
print("Hello, World!")
\end{lstlisting}

\subsection{Exemples illustratifs d’auto-reproducteur}

\subsubsection{Le Quine}

«Un quine\cite{sarkar2020quines} (ou programme auto-reproducteur, self-reproducing en anglais) est un programme informatique qui imprime son propre code source»\footnote{\url{https://fr.wikipedia.org/wiki/Quine_(informatique)}} sans se lire lui-même. Il doit contenir une logique d’écriture de code et contenir ses méta-données de génération. Il est ainsi un générateur de premier niveau.


Voir l'exemple de Code~\ref{lst:exemple_quine}, la sortie textuelle dans le console lors de l'exécution est le même que son code.

\begin{lstlisting}[language=Python, upquote=true, caption={Exemple de code Quine}, label={lst:exemple_quine}]
a='a=%r;print(a%%a)';print(a%a)
\end{lstlisting}

Cependant, le Quine ne sait rien faire d’autre que de s’auto-générer. Ce qui pourrait apporter une contribution serait de faire un auto-reproducteur qui permet de dériver vers d’autres fonctionnalités et ainsi intégrer l’amélioration continue sur son propre développement.

\subsection{Exemples illustratifs de générateur de code}

La génération de code est un des moyens pour supporter le développeur dans son développement d’un logiciel. C’est la partie «création de code» de la méthodologie DevOps.

\subsubsection{Technique de génération de code basique}

Dans le code~\ref{lst:gen_code_basique_c},~\ref{lst:gen_code_basique_uc},~\ref{lst:gen_code_basique_m}, la fonction «eval» en Python est dynamique, c'est-à-dire qu’il permet l’exécution à partir d’une chaîne de caractère, ce type de génération de code ne permet pas une évolution efficace ou une simplification du développement. Ça revient à lire un fichier et à l'imprimer, il n’a pas de capacité dynamique d’adaptation.

\begin{lstlisting}[language=Python, upquote=true, caption={C}, label={lst:gen_code_basique_c}]
print("Hello, World!")
\end{lstlisting}

\begin{lstlisting}[language=Python, upquote=true, caption={µ$_C$}, label={lst:gen_code_basique_uc}]
print('print("Hello, World!")')
\end{lstlisting}

\begin{lstlisting}[language=Python, upquote=true, caption={M(µ$_C$)}, label={lst:gen_code_basique_m}]
eval("""print('print("Hello, World!")')""")
\end{lstlisting}

Dans le code~\ref{lst:gen_code_basique_2_c},~\ref{lst:gen_code_basique_2_uc},~\ref{lst:gen_code_basique_2_m}, cette technique basique est paramétrable, contrairement à la première exemple~\ref{lst:gen_code_basique_m}. Le f-strings fait office de template et il y a la capacité dynamique d’adaptation en ajoutant des itérations et des conditions sans utiliser de bibliothèque externe.

\begin{lstlisting}[language=Python, upquote=true, caption={C - fichier C.py}, label={lst:gen_code_basique_2_c}]
print("Hello, World!")
\end{lstlisting}

\begin{lstlisting}[language=Python, upquote=true, caption={µ$_C$ - fichier uC.json}, label={lst:gen_code_basique_2_uc}]
{
 "fonction": "print",
 "argument": "\"Hello, World!\""
}
\end{lstlisting}

\begin{lstlisting}[language=Python, upquote=true, caption={M(µ$_C$)}, label={lst:gen_code_basique_2_m}]
import json

with open("uC.json") as f:
   metadata = json.load(f)

result = f"{metadata.get('fonction')}({metadata.get('argument')})\n"

with open("C.py", "w") as f:
   f.write(result)
\end{lstlisting}

\subsubsection{Technique de génération de code par template}

Un moteur de template est un outil de modèle structurel qui simplifie la syntaxe pour assurer une bonne maintenabilité et qui est généralement utilisé pour le développement de projet Web.

La génération de code par template est une technique de développement de logiciels qui permet de produire du code source à partir de modèles prédéfinis appelés templates.

Le code~\ref{lst:gen_code_template_c},~\ref{lst:gen_code_template_uc},~\ref{lst:gen_code_template_template},~\ref{lst:gen_code_template_m} utilise la bibliothèque «Jinjer2». C'est un mécanisme similaire à code~\ref{lst:gen_code_basique_2_m}, cependant la logique est intégré directement dans le fichier template.

\begin{lstlisting}[language=Python, upquote=true, caption={C - fichier C.py}, label={lst:gen_code_template_c}]
print("Hello, World!")
\end{lstlisting}

\begin{lstlisting}[language=Python, upquote=true, caption={µ$_C$ - fichier uC.json}, label={lst:gen_code_template_uc}]
{
 "fonction": "print",
 "argument": "\"Hello, World!\""
}
\end{lstlisting}

\begin{lstlisting}[language=Python, upquote=true, caption={template - fichier template}, label={lst:gen_code_template_template}]
{{ fonction }}({{ argument }})
\end{lstlisting}

\begin{lstlisting}[language=Python, upquote=true, caption={M(µ$_C$)}, label={lst:gen_code_template_m}]
import json

from jinja2 import Template

with open("template") as f:
   template = Template(f.read())

with open("uC.json") as f:
   metadata = json.load(f)

result = template.render(
   fonction=metadata.get("fonction"), argument=metadata.get("argument")
)

with open("C.py", "w") as f:
   f.write(result)
\end{lstlisting}

Les avantages du template~\ref{lst:gen_code_template_template_2} pour cette approche sont : une productivité accrue, une réduction des erreurs de codage, une meilleure cohérence du code et une réduction du temps de développement. Cette technique peut également faciliter la maintenance du code, puisque les modifications apportées aux templates sont automatiquement propagées à tout le code généré.

\begin{lstlisting}[language=Python, upquote=true, caption={Exemple de template avec logique}, label={lst:gen_code_template_template_2}]

  <li>
	:):(
	<em>{{ student.name }}:</em> {{ student.score }}/{{ max_score }}
  </li>

\end{lstlisting}

\subsubsection{Générateur de code par template avec Odoo 12}

La technique utilisée par Odoo 12 est le «scaffold», il gère 2 types de template : un module de base «default» et un module thème «theme».

Dans le code~\ref{lst:odoo_scaffold_default}, tout est presque commenté, rien n’est utilisable, mais nous avons la structure \texttt{MVC}. Le gain d’accélération au développement est minime.

\begin{lstlisting}[language=Python, upquote=true, caption={Commande Odoo pour générer un module avec le Scaffold}, label={lst:odoo_scaffold_default}]
> ./odoo/odoo-bin scaffold module_default ./
> tree module_default/
controllers
    controllers.py
    __init__.py
demo
    demo.xml
__init__.py
__manifest__.py
models
    __init__.py
    models.py
security
    ir.model.access.csv
views
    templates.xml
	views.xml
\end{lstlisting}

Cette fois-ci, dans le code~\ref{lst:odoo_scaffold_theme}, tout est vide, le module «theme\_module» fait une erreur à l'installation. Il est plus efficace copier et modifier un thème existant que d'utiliser le «scaffold».

\begin{lstlisting}[language=Python, upquote=true, caption={Commande Odoo pour générer un thème avec le Scaffold}, label={lst:odoo_scaffold_theme}]
> /odoo/odoo-bin scaffold -t theme theme_module ./
> tree theme_module/
demo
    pages.xml
__init__.py
__manifest__.py
static
    src
        scss
            custom.scss
views
    options.xml
    snippets.xml
\end{lstlisting}

Le gain d’accélération au développement est minime, tellement que c’est négligeable. Copier un module fonctionnel et le modifier en enlevant ce qu’on ne veut pas est plus efficace que d’utiliser cette technique. Disons que cette technique est utile pour un débutant pour comprendre à quoi ressemble une architecture vide, mais il va apprendre plus vite en regardant le fonctionnement de module fonctionnel.

\subsubsection{Technique de génération de code par rétro-ingénierie}

Lorsqu'on veut faire de la rétro-ingénierie\footnote{\url{https://fr.wikipedia.org/wiki/Rétro-ingénierie_en_informatique}} avec un générateur le code, l'objectif est de faire de la ré-ingénierie sur la partie génération, ainsi on altère le code, voir figure~\ref{fig:retro_re_ing}. %\footnote{Modification de l'image source : \href{https://fr.wikipedia.org/wiki/Rétro-ingénierie_en_informatique#/media/Fichier:Retroingenierie_-_Byrne.svg}}.

\begin{figure}[htb]
\centering
\includegraphics[width=5in]{images/Retroingenierie_-_Byrne.png}
\caption{Altération du code avec la rétro-ingénierie et la ré-ingénierie.}
\label{fig:retro_re_ing}
\end{figure}

% Exemple
% Texte en \emph{italique}, \textsc{petites majuscules}, mot \mbox{insécable}.\\
% Texte \ul{souligné}, \hl{surligné}, \textbf{gras}.\\
% Texte entre ``guillemets''.\\
% Police \texttt{monospace}.\\
% Un mot courant en réseautique mobile: n\oe{}ud\footnote{Note de bas de page.}.\\
% L'objet RSVP \texttt{SENDER\_TEMPLATE}.\\
% Nom d'un auteur: \citeauthor{RFC_IPv4}.\\
% Une architecture 32~bits.\\
%%
%%  CONCEPTS DE BASE / BASIC CONCEPTS
%%
% \section{Définitions et concepts de base}  % environ 2-3 pages
% \begin{flushleft}
% 1\iere{} utilisation d'un acronyme yeah: \ac{IETF}.\\
% 2\ieme{} utilisation d'un acronyme: \ac{IETF}.\\ ça ne marche pas
% Acronyme au long: \acl{IETF}.\\
% \end{flushleft}

% \subsection{Une sous-section}
% Un URL: \href{http://www.polymtl.ca}{École Polytechnique de Montréal}.

% \subsubsection{Une sous-sous-section}
% Les besoins des flots de données peuvent être catégorisés selon
% quatre paramètres importants \cite{Fraas2010} ou:
% \begin{itemize}
% \item la fiabilité (acheminement des données avec succès)~;
% \item le délai de \mbox{bout-en-bout} de la source vers la destination~;
% \item la variation du délai de \mbox{bout-en-bout} (\emph{jitter})~;
% \item la bande passante requise (le débit des informations).
% \end{itemize}

% \paragraph{Le niveau paragraphe} est plus bas encore dans la hiérarchie\ldots
% Une citation entre parenthèses \cite{SAHIN2020}.
% ou des citations entre parenthèses \cite{Haist2014,Senjian2015,Madani2010}.

% \clearpage

% %%
% %% ELEMENTS DE LA PROBLEMATIQUE
% %%
% \section{Éléments de la problématique}  % environ 3 pages
% La description de \mbox{l'en-tête} commun de RSVP est détaillée ci-dessous:\\
% \begin{tabular}{p{1in}p{4.5in}}
% &\\ % Ligne vide
% \texttt{Ver}: & \texttt{4 bits}\\
%           & Version du protocole. La version actuelle est~1.\\[5pt]
% \texttt{Flags}: & \texttt{4 bits}\\
%           & Aucun Flag n'est défini. L'émetteur doit (\textbf{MUST})
%           mettre le champ à zéro et le récepteur doit (\textbf{MUST})
%           ignorer ce champ.\\[5pt]
% \texttt{Msg Type}: & \texttt{8 bits}\\
%           & Type de message\\[5pt]
% \texttt{Checksum}: & \texttt{16 bits}\\
%           & Complément à un du complément à un de la somme des champs
%           de \mbox{l'en-tête}, avec le champ Checksum à~0 pour des
%           fins de calcul. La valeur~0 signifie qu'aucun Checksum n'a
%           été transmis. Si le résultat du calcul du Checksum donne~0,
%           la valeur 0xFFFF doit être stockée dans ce champ.\\[5pt]
% \texttt{TTL}: & \texttt{8 bits}\\
%           & Valeur originelle du champ \texttt{TTL} utilisée pour
%           transmettre ce message.\\[5pt]
% \texttt{Reserved}: & \texttt{8 bits}\\
%           & Réservé pour usage futur. L'émetteur doit (\textbf{MUST})
%           mettre le champ à zéro et le récepteur doit (\textbf{MUST})
%           ignorer ce champ.\\[5pt]
% \texttt{Length}: & \texttt{16 bits}\\
%           & Longueur totale du message en octets, incluant
%           \mbox{l'en-tête} commun et tous les objets de longueur
%           variable.
% \end{tabular}

% \subsection{Autres types de structures de données}
% L'énumération:
% \begin{enumerate}
% \item Un item~;
% \item Un autre item.
% \end{enumerate}


% \subsection{Le protocole IPv6}
% Voir la Figure~\ref{fig:IPv6} pour plus de détails. Le champs DSCP est
% décrit dans le Tableau~\ref{tab:RangesDSCP}.

% \begin{figure}[htb]
% % [htb] place la figure ici + en haut ou en bas de la page. 
% % [htb] places the figure here + top or bottom of the page. 
% % Vous pouvez également utiliser [tb] pour placer les figures en haut ou en bas de la page et [p] pour les placer sur une page ne contenant que des flottants (ex. : tableaux, figures).
% % You can also use [tb] for placing figures on the top or the bottom of a page and [p] for a figure placed on a page containing only floats (ex.: tables, figures).
% % Plus d'informations / More information here: https://www.ctan.org/tex-archive/info/epslatex/english 
% \centering
% \includegraphics[width=4in]{IPv6_header}
% \caption{L'en-tête IPv6}
% \label{fig:IPv6}
% \end{figure}

% \begin{table}[htb]
% \caption{Plages de valeurs pour le champ \texttt{DSCP}}
% \centering
% \begin{tabular}{|c|c|l|}
% \hline\rowcolor[gray]{0.8}\color{black}
% Plage & Valeurs & Règle d'assignation\\\hline
% 1 & xxxxx0 & Assignation par une norme de l'IANA\\\hline
% 2 & xxxx11 & Expérimentation/Usage local\\\hline
% 3 & xxxx01 & Expérimentation/Usage local (pourrait être jointe à la plage 1)\\\hline
% \end{tabular}
% \label{tab:RangesDSCP}
% \end{table}

% % On veut éviter que la figure et le tableau soient placés au-delà de la section courante.
% % To prevent the figure and table from being positioned outside of the current section. 
% \FloatBarrier


% %%
% %% OBJECTIFS DE RECHERCHE / RESEARCH OBJECTIVES
% %%
% \section{Objectifs de recherche}  % 0.5 page
% Les objectifs de la recherche sont de concevoir un algorithme $O(n)$.


% %%
% %% PLAN DU MEMOIRE / THESIS OUTLINE
% %%
% \section{Plan du mémoire}  % 0.5 page

% Voir la Figure~\ref{fig:Layers} pour plus de détails. 

% \begin{figure}[htb]
% \centering
% \includegraphics[width=4in]{demo_tikz}
% \caption{Couches}
% \label{fig:Layers}
% \end{figure}


% Un tableau : / A table:
% \begin{table}[htb]
%   \centering
%   \caption{Constantes et variables du modèle analytique}
%   \begin{tabular}{|c|l|}
%     \hline\rowcolor[gray]{0.8}\color{black}
%     Symbole         & Description\\\hline
%     $\lambda$       & Taux d'arrivée moyen des requêtes de réservation de ressources\\\hline
%     $\frac{1}{\mu}$ & Durée moyenne d'une session\\\hline
%     $C$             & Capacité d'une cellule (nombre de sessions supportées)\\\hline
%     $v_{moy}$       & Vitesse moyenne des MN dans le réseau d'accès\\\hline
%     $L$             & Longueur d'un côté d'une cellule carrée\\\hline
%     $n$             & Nombre moyen de MN dans une cellule\\\hline
%     $\rho$          & Charge d'une cellule\\\hline
%     $P_b$           & Probabilité de blocage d'une requête de réservation\\\hline
%     $P_f$           & Probabilité d'interruption forcée d'une session\\\hline
%     $P_c$           & Probabilité de compléter une session avec succès\\\hline
%     $\Delta{}T$     & Délai de transmission\\\hline
%   \end{tabular}
%   \label{tab:Definitions}
% \end{table}

% La formule d'\mbox{Erlang-B}:
% \begin{equation}
%   P_b = \frac{\frac{\rho^C}{C!}}{\sum\limits_{x=0}^{C}\frac{\rho^x}{x!}}
%   \label{eq:Pblock}
% \end{equation}

% Une autre équation : / Another equation:
% \begin{equation}
%   \begin{split}
%     P_c &= (1 - P_b) \times (1 -  P_f)^N\\
%         &= (1 - P_b)^{N+1}
%   \end{split}
%   \label{eq:ProbComplete}
% \end{equation}

% Enfin, l'expression suivante indique le moment à partir duquel les
% réservations de ressources sont en place:
% \begin{equation}
%   \Delta{}T_{init} =
%   \begin{cases}
%     2\Delta{}T_{E2E} & \Delta{}T_{wan} > (\Delta{}T_{rad} + \Delta{}T_{net})\\
%     \Delta{}T_{E2E} + 3(\Delta{}T_{rad} + \Delta{}T_{net}) & \text{sinon}
%   \end{cases}
%   \label{eq:InitCost}
% \end{equation}

% \paragraph{Le taux de paquets perdus} correspond au nombre de paquets
% éliminés à cause d'une erreur de \emph{checksum} à un n\oe{}ud
% quelconque ou d'une situation de congestion. Le taux de paquets perdus
% pour un chemin est déterminé de la façon suivante:
% \begin{equation}
%   \label{eq:genPLR}
%   PLR_P = 1 - \prod_{i=1}^N(1 - PLR_i)
% \end{equation}

% Toutefois, si les taux d'erreurs sont très faibles, comme c'est
% généralement le cas pour des liens optiques, on peut approximer
% $PLR_P$ de façon à le transformer en un paramètre additif:
% \begin{equation}
%   \label{eq:approxPLR}
%   \begin{split}
%     PLR_{L_1 \oplus L_2} &= 1 - (1 - PLR_1)(1 - PLR_2)\\
%     &= 1 - (1 - PLR_2 - PLR_1 + \underbrace{PLR_1
%       \times PLR_2}_\text{négligeable})\qquad PLR_1 \ll 1,
%     PLR_2 \ll 1\\
%     &\approx PLR_1 + PLR_2
%   \end{split}
% \end{equation}

% \clearpage

% Une courbe : / A curve:
% \begin{figure}[htb]
% \centering
% \includegraphics[width=5in]{LinkUsage}
% \caption{Délai moyen en fonction du taux d'utilisation d'un lien}
% \label{fig:LinkUse}
% \end{figure}

% \selectlanguage{english}
% This paragraph is formatted by \LaTeX{} according to the standard rules of the
% English language (\mbox{e.g.} hyphenation).
% \selectlanguage{french}

% L'arithmétique en virgule flottante peut entraîner des erreurs
% d'approximation et il est important d'en être conscient
% \cite{Rossi2011}.

% De même, les calculs effectués sur une carte graphique (GPU) peuvent
% introduire des erreurs d'approximation \cite{DeSantis2002, Cohen2006,
%   Thorsson2014, Schirmer2012, Sakai2015, Electrical2006,
%   Min2016, Massicotte2013, Kaliouby1987, Daintith2010, Haist2014, Kizza2013,
%   Manasreh2011, Brydson1999, Boyce2002}.
