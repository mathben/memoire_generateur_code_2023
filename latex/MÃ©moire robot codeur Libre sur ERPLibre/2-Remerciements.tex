% Remerciements / Acknowledgements
%
% Grâce aux remerciements, l'auteur attire l'attention du 
% lecteur sur l'aide que certaines personnes lui ont apportée, 
% sur leurs conseils ou sur toute autre forme de contribution 
% lors de la réalisation de son mémoire ou thèse. Le cas 
% échéant, c'est dans cette section que le candidat doit 
% témoigner sa reconnaissance à son directeur de recherche, aux 
% organismes dispensateurs de subventions ou aux entreprises qui
% lui ont accordé des bourses ou des fonds de recherche.

% Through the acknowledgements, the author draws the
% reader's attention to the help that certain people 
% have given them, their advice or any other form of 
% contribution during the completion of the 
% dissertation or thesis. If applicable, it is in 
% this section the candidate should acknowledge the 
% assistance of their advisor, granting agencies or 
% companies that have provided research grants or
% funds.
\ifthenelse{\equal{\Langue}{english}}{
	\chapter*{ACKNOWLEDGEMENTS}\thispagestyle{headings}
	\addcontentsline{toc}{compteur}{ACKNOWLEDGEMENTS}
}{
	\chapter*{REMERCIEMENTS}\thispagestyle{headings}
	\addcontentsline{toc}{compteur}{REMERCIEMENTS}
}
%

Je tiens à remercier mon directeur de recherche en génie industriel, Samuel Basseto, qui m'a aiguillé vers le projet en lien avec l'amélioration continue en contexte industriel et qui a fait la liaison avec l'Accorderie, ainsi qu'avec des projets similaires. Je veux exprimer aussi ma gratitude envers les membres de son laboratoire de recherche, LABAC, pour tous les commentaires pertinents, les réunions et les moments plus décontractés et, en particulier, Lucas Jacquet qui a choisi de faire un projet connexe au mien.

Merci à l'Accorderie et sa directrice, Nadia Mohamed-Azizi, d'avoir permis que l'organisation devienne un projet d'étude. Dans le même ordre d'idée, merci à Geneviève David et Camille Bolduc-Boyer du Centre d'excellence sur le partenariat avec les patients et le public du CR-CHUM pour leur confiance et leur participation dans ce qui est devenu le deuxième cas du projet d'étude.

Je voudrais souligner l'apport financier de la Fondation Trottier, ainsi que l'investissement de la compagnie TechnoLibre en salaire pour l'avancement et la réalisation du projet ORE.

Un grand merci à mes relecteurs :  Alexandre Benoit pour son travail sur la forme, Hassan Kassi pour son mentorat et ses idées et Simon de Montigny pour son soutien, pour le peaufinage de la présentation des résultats et l'amélioration de la cohérence de ces derniers.

Un grand merci à mes parents, Sylvie Lemieux et Sylvain Benoit, pour avoir cru en moi et m'avoir donné leur soutien financier et moral, non seulement pour ma maîtrise, mais tout au long de mes études.

Un merci spécial à ma conjointe, Marie-Michèle Poulin, pour son appui indéfectible tout au long du projet, non seulement dans les trivialités de la vie courante, mais grâce à ses idées, ses questionnements et ses connaissances, ainsi qu'à son apport au document au point de vue de la révision du langage et la traduction. 

Enfin, un merci à mon bébé, Elizabeth Benoit, qui me donne la motivation d'avancer et qui me donne espoir qu'elle veuille créer des communautés et qu'elle ait envie de vivre libre et selon ses propres choix.

% \textbf{Samuel Bassetto}

% Directeur de recherche en génie industriel, aide à l’amélioration continue en contexte industriel, aide dans la création de lien avec le projet d’étude de l’Accorderie et aux projets similaires.

% \textbf{Alexandre Benoit}

% Relecture

% % \textbf{Célia Lignon}

% % Pour la maquette du projet Espace Membre Accorderie fait en collaboration avec DOMUS de l’université de Sherbrooke.

% \textbf{Centre d'excellence sur le partenariat avec les patients et le public}

% Geneviève David
% Camille Bolduc-Boyer

% Projet d’étude 2

% \textbf{labac}

% \textbf{Fondation Trottier}

% Financement

% \textbf{Hassan Kassi}

% Relecture

% \textbf{Marie-Michèle Poulin, marieplume}

% Relecture

% \textbf{Nadia Mohamed-Azizi}

% Directrice du Réseau Accorderie.

% \textbf{Réseau Accorderie}

% Projet d’étude 1


% \textbf{Simon Montigny}

% Relecture

% \textbf{TechnoLibre}
% % TODO doit exprimer ORE ailleurs dans le texte
% Prêt d’équipement et d'investissement en salaire pour avancer le projet Offrir Recevoir Echanger pour le projet d’étude.

% \textbf{Elizabeth Benoit}

% Bonne envie de vivre libre par ses propres choix, combien de communauté voudra-t-elle créer?
