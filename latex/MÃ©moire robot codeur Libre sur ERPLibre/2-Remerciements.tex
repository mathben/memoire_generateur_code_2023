% Remerciements / Acknowledgements
%
% Grâce aux remerciements, l'auteur attire l'attention du 
% lecteur sur l'aide que certaines personnes lui ont apportée, 
% sur leurs conseils ou sur toute autre forme de contribution 
% lors de la réalisation de son mémoire ou thèse. Le cas 
% échéant, c'est dans cette section que le candidat doit 
% témoigner sa reconnaissance à son directeur de recherche, aux 
% organismes dispensateurs de subventions ou aux entreprises qui
% lui ont accordé des bourses ou des fonds de recherche.

% Through the acknowledgements, the author draws the
% reader's attention to the help that certain people 
% have given them, their advice or any other form of 
% contribution during the completion of the 
% dissertation or thesis. If applicable, it is in 
% this section the candidate should acknowledge the 
% assistance of their advisor, granting agencies or 
% companies that have provided research grants or
% funds.
\ifthenelse{\equal{\Langue}{english}}{
	\chapter*{ACKNOWLEDGEMENTS}\thispagestyle{headings}
	\addcontentsline{toc}{compteur}{ACKNOWLEDGEMENTS}
}{
	\chapter*{REMERCIEMENTS}\thispagestyle{headings}
	\addcontentsline{toc}{compteur}{REMERCIEMENTS}
}
%

\textbf{Samuel Bassetto}

Directeur de recherche en génie industriel, aide à l’amélioration continue en contexte industriel, aide dans la création de lien avec le projet d’étude de l’Accorderie et aux projets similaires.

\textbf{Alexandre Benoit}

Relecture

\textbf{Célia Lignon}

Pour la maquette du projet Espace Membre Accorderie fait en collaboration avec DOMUS de l’université de Sherbrooke

\textbf{Centre d'excellence sur le partenariat avec les patients et le public}

Projet d’étude 2

\textbf{Fondation Trottier}

Financement

\textbf{Hassan Kassi}

Relecture

\textbf{Marie-Michèle Poulin}

Relecture

\textbf{Réseau Accorderie}

Projet d’étude 1

\textbf{Simon Montigny}

Relecture

\textbf{TechnoLibre}

Prêt d’équipement et d'investissement en salaire pour avancer le projet ORE pour le projet d’étude
