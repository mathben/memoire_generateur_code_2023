\Chapter{OBJECTIFS ET MÉTHODOLOGIE}\label{sec:Theme1} \label{chapitre_methode}
Dans ce chapitre, nous examinons l’objectif général de ce projet qui a été de créer et de valider le fonctionnement d’un robot logiciel générateur de code qui sert à l’implantation d’un \textit{ERP} libre. Plus spécifiquement, nous allons nous concentrer sur cinq sous-objectifs (SO) suivants, soit: développer un générateur de code de module sur Odoo; développer une logique d'amélioration continue sur l'écriture du code; développer une interface permettant de paramétrer la génération de code;
développer un système de distribution et développer un système de gestion de communauté.

La méthodologie classique en recherche scientifique, voir l'annexe~\ref{annexe_methodologie}, est représentée par la section à droite qui débute par «Faire une revue de littérature», elle est bonne en théorie. Cependant, au niveau pratique, nous avons adopté la méthodologie Agile\footnote{\url{https://fr.wikipedia.org/wiki/M\%C3\%A9thode_agile}} pour le développement des composantes d'un générateur de code, représentées par la section de gauche du schéma en annexe~\ref{annexe_methodologie}. Les fonctionnalités de génération de code ont été validées par des tests de comparaison entre les codes générés et les codes révisés par le développeur. Dans les recherches de solution existante, le générateur de code développé par Luis en 2019~\cite{bluiksnot_repo} a été publié sur Github, cependant il a nécessité des modifications qui seront détaillées dans les résultats. 

\section{SO-1 - Générateur}
Cette section concerne la génération de code, la génération de module et génération de modèles de données. Les étapes à accomplir sont de développer une logique d’écriture de module sur une architecture de MVC avec support de plateforme web, la mise en place d’un concept de gabarit de code qui génère du code, la mise en place de la génération de code à partir de données et la génération d'un module à partir d’une base de données externe.

\section{SO-2 - Rétro-ingénierie}
Cette section porte sur la partie de rétro-ingénierie, l'amélioration continue et la qualité logicielle. Les étapes à accomplir sont de développer la capacité de comprendre une structure de code et de la reproduire, de définir ce qui est de l’amélioration continue et son application dans un contexte d’automatisation, de mettre en place des tests de validation de code sur des critères de qualité mesurables et d'intégrer des règles de codage standardisées pour favoriser le réseau d’entraide.

\section{SO-3 - Interface}
Cette section concerne l'interface utilisateur. Les étapes à accomplir sont de proposer une classification des techniques que le robot logiciel codeur peut réaliser en programmation, de développer une interface permettant le contrôle du robot logiciel codeur pour l’orienter dans la programmation de fonctionnalités et de rendre disponible une interface LCNC pour permettre aux utilisateurs de programmer leurs fonctionnalités.

\section{SO-4 - Déploiement}
Pour la section déploiement, il faut développer un système de distribution du robot logiciel codeur.

\section{SO-5 - Réseau d’entraide}
Cette section comporte la gestion du développeur, ainsi que les projets d'études. Les étapes à accomplir sont de documenter les processus de développement pour amener les utilisateurs à contribuer et les faire participer à la maintenance, de mettre des guides pour permettre le sentiment d'appartenance, de mettre en place une politique tolérance zéro avec un système de communication non violente et créer un lieu de discussion publique, d'élaboration du prototype pour les spécifications du réseau de l’Accorderie du Canada et d'élaboration du prototype pour les spécifications de l’organisme CEPPP du Canada.

\section{Environnement informatique}
L'environnement informatique choisi est la plateforme ERPLibre 1.5.0 contenant Odoo 12 communauté, qui utilise les langages Python 3.7, XML, Javascript et SCSS. Nos tests et développements ont été effectués sur le système d’exploitation Ubuntu 20.04.
Pour les temps d’exécution des tests, ils ont été effectués sur une machine avec le \textit{CPU AMD Ryzen} 9 5950X, mémoire \textit{ram} 2x«32GiB DIMM DDR4 \textit{Synchronous Unbuffered (Unregistered)} 2667 MHz (0.4 ns)», et disque 1To wd-black-sn850-nvme.

\section{Méthodologie de test}
Les tests ont été codés directement dans ERPLibre, dans un script Python, avec un ensemble de scripts pour valider les différences dans le Git, après exécution et cacher des différences.

Le tout a été programmé grâce à la technique de parallélisme avec la bibliothèque \textit{Asyncio} et des nouveaux processus et non des \textit{thread}\footnote{Processus légé}. Seul le test de nouveau projet est exécuté après le parallélisme, ce qui ajoute plus de temps à l’exécution.

Puisque le test d’un générateur de code consiste à valider ce qu’il a généré, vérifier la couverture de code est un bon indicateur pour déterminer le code utilisé et valider ce qui est déprécié dans le générateur.

\subsection{Couverture du code}
La couverture du code a été faite avec la bibliothèque \textit{Coverage} version 7.0.1 en Python et configurée sur le répertoire qui contient le générateur de code pour faire le suivi des lignes exécutées dans la machine. Ça devient une métrique de la performance d’utilisation sur la quantité de fonctionnalités générées.
