\Chapter{CONCLUSION}\label{sec:Conclusion}
% Texte / Text.

%%
%%  SYNTHESE DES TRAVAUX / SUMMARY OF WORKS
%%
\section{Synthèse des travaux}
% Texte / Text.
Les résultats obtenus ont permis d’atteindre en tout ou en partie l’ensemble des sous-objectifs énoncés dans le chapitre~\ref{chapitre_methode}.%, voir Tableau~\ref{tab:synthese_travaux}.

% TODO synthèse du générateur de code

\subsection{Générateur de code}

En partant de modules existants, le développement du générateur de code a été orienté pour avoir une traçabilité sur les données de génération des modules. Puis des techniques de rétro-ingénierie ont été implémentées pour permettre l'extraction d'information et intégrer des boucles d'amélioration continue sur le développement, permettant la génération de code automatisée. Un mécanisme de test de génération de code a été implémenté pour éviter la régression sur le développement. Le développement de module dans Odoo 12 est maintenant accéléré pour plusieurs contextes.

\subsection{Gestion de communauté autour de projet libre}
Pour la gestion de communauté autour de projet libre, 3 guides ont été rédigés pour permettre d'outiller le gestionnaire de communauté à intégrer des participants dans leur développement technologique. Un guide pour démarrer un projet, un guide pour gérer les participants et un guide pour les règles éthiques d'hébergement.

\subsection{Projet Accorderie}

Une analyse des besoins fonctionnels a été effectuée pour comprendre les développements à réaliser. La migration de la base de données d'une plateforme en PHP a été réussie en utilisant le générateur de code, mais elle est encore, à ce jour, en adaptation vers un modèle Odoo plus intégré au ERP. Les efforts ont été mis pour la création d’une interface utilisateur avec des technologies qui n’étaient pas, à la base, supportées dans Odoo.

\subsection{Projet Portail CEPPP}

Une analyse des besoins fonctionnels a été effectuée pour comprendre le développement à réaliser. Le générateur de code a été utilisé en début de projet pour faire la migration des fonctionnalités personnalisées par un ancien développeur sur la plateforme SuiteCRM. Cela a permis d'économiser du temps de développement et de réduire les erreurs possibles de retranscription du langage PHP au langage Python, ainsi que la génération des vues administrations et portails. Cependant, nous avons cessé d'utiliser le générateur de code suite à la ré-ingénierie, dès que le projet a commencé à diverger vers des fonctionnalités personnalisées qu’il ne pouvait plus supporter, telle que l'anonymisation des données. La nouvelle solution est utilisée en production.

%%
%%  LIMITATIONS
%%
\section{Limitations de la solution proposée}\label{sec:Limitations}
% TODO dire en quoi ce que tu as fait améliore l’état de l’art décrit dans la littérature SYNTHÈSE
% TODO décrire les limitations/faiblesses de ce que tu as produit comme logiciel/résultats LIMITATION

Les limitations sont causées, en général, par le manque de temps de terminer les développements et seulement la ligne critique a été suivie pour réaliser la liste des tâches nécessaires à la réponse des demandes-client.

\subsection{Couverture des tests}
Les tests devraient couvrir 100\% du code, cependant la couverture est de 84\% pour 3 raisons :

\begin{enumerate}
    \item Il y a du code fonctionnel non testé, il manque des tests;
    \item Il y a du code désuet qu’il faut nettoyer ou refactoriser;
    \item La gestion des erreurs n’est pas couverte, il faudrait les ignorer dans le test de couverture et faire des tests unitaires qui valident la gestion des erreurs.
\end{enumerate}

\subsection{Personnalisation du développement}
La personnalisation de module ERP passe par le nombre de technique supporté, mais aussi par la paramétrisation de ces techniques. L'accent du projet a été mis sur le support des techniques et sur permettre l'amélioration continue sur des modules. Toutes les combinaisons n'ont pas été testées et il manque plusieurs possibilités de paramétrisation.

\subsection{Auto-reproduction du générateur de code}
%todomarie cette phrase n'est vraiment pas claire, tu devrais mieux expliquer - réponse ajouter de la clarté
Bien qu'il y ait du progrès dans l'auto-reproduction du générateur de code, cet objectif n'est pas complété. Le générateur de code ne supporte pas encore toutes les techniques et combinaisons que celui-ci nécessite pour fonctionner. Les travaux doivent continuer sur cette ligne critique en améliorant le générateur de code jusqu'à ce qu'il soit apte à s'auto-reproduire.

\subsection{Extraction de code et de modèles de base de données}
L'extraction de code PHP, Javascript ou de modèles de base de données a été appliqué sur des contextes particuliers tels que le projet Accorderie et CEPPP. Ainsi, il sera nécessaire de faire des ajustements pour supporter de nouveaux projets et s'assurer de mettre ces techniques d'extraction dans les tests de régression.

%%
%%  AMELIORATIONS FUTURES / FUTURE RESEARCH
%%
\section{Améliorations futures}
% Texte / Text.
% TODO dire ce qui doit être fait pour que les aspects d’utilisabilité de ton logiciel en terme de communauté, de libre, etc soient complets.

% TODO parler de la sympoïèse comme étant la suite sur la partie communautaire et la distribution de système
% TODO regarder comment intégrer du «Model Driven Architecture»

%todomarie je ne comprends pas du tout cette phrase/section
Ce projet pourrait avoir une incidence beaucoup plus importante en prenant en considération les apports futurs sur les 5 thématiques suivantes soit l'amélioration du générateur de code, le projet Accorderie, le projet Ceppp, le projet NLP et le projet de support de développement de module dans la communauté Odoo.

\subsection{Amélioration du générateur de code}

\subsubsection{Amélioration des instances clients}

Il faut intégrer la génération de code à l’intérieur des instances clients dans l’objectif de la rendre accessible au gestionnaire de déploiement pour y ajouter les nouvelles fonctionnalités suivantes : démarrer la mise à jour, les tests, les améliorations, la migration, l'importation. Les instances clients devraient proposer aux clients via les interfaces l'ajout ou le retrait de fonctionnalités selon leurs retours d'expérience et leurs habitudes d'utilisation.

% Intégration de plusieurs types de réseaux de neurones accessibles librement, il doit être compatible avec le libre.

% Il faut suggérer aux clients des améliorations et de communiquer avec leurs gestionnaires de déploiements.

% Un générateur de code a été créé. L’embryon est créé, il faut terminer sa génération du générateur de code. Les déviances sont déjà créées

\subsubsection{Amélioration de l'auto-génération}

Une fois que le générateur de code aura atteint 100\% d’auto-génération, il restera limité à ne produire que les fonctionnalités qu’il utilise. Il faut faire des tests pour les fonctionnalités qu’il n’utilise pas (ou les combinaisons non utilisées) pour se reproduire. Il reste à auto-générer toutes ses techniques dans des modules qui font de l'héritage sur le générateur de code. L'auto-génération complète va ouvrir la possibilité à une restructuration de l'architecture par le développeur assistée du générateur pour faciliter l'ajout de techniques d'ingénierie dans la génération de code.

% Ajout de la génération de test de base, génération de documentation, mise à jour, migration lors d’une mise à jour sur les données, migration d’une mise à jour de la plateforme.

\subsubsection{Amélioration de l’architecture}

Voici un survol des améliorations de l'architecture possibles : Parallélisation de tout le code, en tout temps, lorsque cela est possible; Automatisation de la configuration pour le déverminage; Automatiser la détection des anomalies; Améliorer l’interface LCNC pour pouvoir accomplir les mêmes étapes que le mode de paramétrisation «Code hook»; Implémenter de l'auto-ingénierie sur le développement des techniques du générateur de code et Génération de code par requis fonctionnel.

De plus, il faudrait supporter de nouvelles architectures dans la génération de code comme : des applications Cordova pour le support mobile natif;  des extensions Javascript dans Gnome Shell pour étendre les fonctionnalités du ERP directement sur le bureau d’un ordinateur sans passer par un navigateur web; d'autres modules sur différents systèmes ERP tels que Tryton ou NextERP; générer des scripts de développement dans le projet ERPLibre qui ne dépendent pas d’Odoo; supporter des applications embarquées pour contrôler des automates ou des systèmes de production automatisée.

% Problème d’extraction, il était dans le générateur de code au départ dans le développement, il y a donc une extraction à deux endroits qui rend complexe la gestion du code. Au fil de la progression du développement, l’extraction de données par rétro-ingénierie s’avère plus efficace que les méta-données du module dans le système Odoo.

% L’auto-ingénierie sur la machine est en cours d’exécution sur le module de base, il faudrait aussi supporter les modules hérités qui sont définis comme des techniques.

% Découper les fonctionnalités d'extraction et de génération, puis les séparer dans des modules des techniques du générateur de code.

% Il faudrait réduire le nombre de techniques dans Base pour qu’ils soient des modules externes par technique, ça faciliterait le changement d’architecture sur la gestion des méta-données, à adapter selon la rétro/ré-ingénierie.

% TODO mettre photo d'amélioration architecture

Ces travaux d’amélioration devront être effectués après l’auto-génération complète des modules Odoo de génération de module.

\subsubsection{Amélioration de la gestion de projet et statistiques}

Le générateur de code doit offrir des outils de gestionnaire de projet pour suivre le développement, faire la liaison entre les demandes clients et les avancements des développeurs, le suivi de la ré-ingénierie des processus d'affaires de l'entreprise, etc. Bref, il doit permettre de faire le suivi des étapes des facteurs décisionnels pour l'implantation d'un ERP.

De plus, puisque l’état des méta-données évolue, il devient difficile de faire le suivi des performances du générateur de code, puisqu’il vient aider dans les boucles d’itérations. Ainsi, il faudrait dégager des statistiques sur ces itérations pour évaluer la contribution du générateur versus l'implication du développeur.

% Il manque l’analyse des différences de code sur les différentes sections générées.

\subsubsection{Suite du développement du générateur de code}

Autre que la modification de l'architecture, le générateur de code doit générer des tests fonctionnels sur les modules Odoo, générer de la documentation fonctionnelle et technique et développer de la migration de module et de données.

% selon les changements du code de versions antérieurs des versions antérieurs. Il faudrait aussi un suivi des fonctionnalités selon les exigences clientes.

Une fois l'auto-reproduction complétée, la prochaine étape est de tester sa mise à niveau de tous les modules dans la communauté et détecter les techniques manquantes par supervision du développeur pour les implémenter. Une fois que tous les modules de la communauté seront gérés, nous pourrons implémenter la migration vers des mises à jour de la plateforme, c’est-à-dire passer d'odoo 12 vers Odoo 14, puis vers Odoo 16. La migration va nécessiter l'adaptation des techniques par différentes version de la plateforme.

% Une fois que l’auto-poïèse sera en place sur la gestion de la machine, la prochaine étape sera de faire l’auto-poïèse sur tout le code Odoo pour le développement de l’architecture.

\subsection{Projet Accorderie}

Une plateforme de l'espace membre est en finalisation de développement, il faudrait poursuivre la mise à jour du générateur de code pour supporter cette nouvelle technologie pour des projets futurs similaires.

% TODO terminé la plateforme Accorderie avec génération AngularJS ou Angular

\subsection{Projet Portail CEPPP}

Le générateur de code doit être en mesure de produire les personnalisations nécessaires au projet Portail CEPPP, telles que l'anonymisation des données. Lorsque ces fonctionnalités seront supportées, la prochaine étape est d'amener le client à utiliser par lui-même le générateur de code pour la maintenance future de la plateforme. Ainsi, nous pourrons évaluer la facilité d'utilisation pour des utilisateurs non développeur.

\subsection{NLP}
%todomarie est-ce que NLP est défini qqpart? - réponse tous les accronymes sont définis dans le fichier Sigles_Abrev.tex
La technologie NLP va permettre de comprendre des textes rédigés par l’utilisateur et les associer à des techniques de programmation. Le NLP est une solution alternative pour interfacer avec l’utilisateur et communiquer avec ce dernier pour développer des logiciels. De plus, cela va améliorer la rédaction de documentation adaptée aux contextes de l'utilisateur. 

Il existe la communauté Hugging Face qui base ses technologies d'apprentissage automatique sur des solutions ouvertes. Il existe des solutions NLP accessibles et compatibles avec le générateur de code.

% TODO Définir le besoin 
% TODO parler d’émotion et empathie

% Qu’est-ce qu’il faudrait développer pour se rendre au stade automate codeur?

% Suggestion d’explorer l’outil libre : Utiliser «HuggingFace» qui contient une grande communauté autour du développement d’un réseau de neurones pour faire du NLP par exemple, c’est compatible avec le logiciel libre.

% \subsection{Support de développement de module dans la communauté Odoo}

% ERPLibre supporte Odoo 12, puisque c’est lui qui supporte le plus de modules. Cependant, ces données représentent seulement celles des repos utilisés par ERPLibre, il y en a beaucoup plus dans la communauté. C’est pourquoi il faut faire une recherche de ces modules dans la communauté et entreprises, et les rendre accessible par une base de données publiques~\footnote{Comme fait sur \url{https://odoo-code-search.com/}}.

% Selon l’évolution, il faudrait migrer vers la version 14.0 en 2024. Donc, il va falloir supporter la migration de modules vers des versions supérieures.




%%
%%  CONCLUSION
%%

% Conclusion de la conclusion
% Nous avons besoin de continuer à faire de la recherche, pleins de question n'a pas été répondu, comment arriver à mettre en place une technopoïèse avec un automate générateur de code qui sait appliquer les 4 libertés? Quels seront les performances d'une machine autopoïèse? Va-t-on réussir à créer un robot codeur libre?

% L'autopoïèse qui sera développé servira à produire la représentation du contenu de ma thèse que je poursuis.
\section{Importance de la recherche}
En conclusion, le robot logiciel libre codeur est en première phase de développement incluant la génération de code, l'interface avec l'utilisateur et la rétro-ingénierie pour appliquer de l'amélioration continue orientée au support d'un réseau d'entraide. 

L’automatisation du développement de logiciel va permettre l’accélération de création de fonctionnalités et la réduction des coûts de développement. 

Le robot va permettre aux chercheurs d’être plus efficaces dans leurs travaux en facilitant le développement de leurs propres outils pouvant mieux tracer, s’interfacer et avoir le contrôle de leurs données.

Les communautés auront accès à des fonctionnalités plus aisément, permettant ainsi l'adaptation à des états d'urgence, grâce, par exemple, au partage de banques de temps de service en temps de crise. 
Avec les crises mondiales, les états d’urgence seront de plus en plus présents, l’accélération du développement technologique aura un impact sur l’appropriation de la technologie des communautés pour leur permettre de mettre en place des solutions rapidement.