\Chapter{CONCLUSION}\label{sec:Conclusion}
% Texte / Text.

%%
%%  SYNTHESE DES TRAVAUX / SUMMARY OF WORKS
%%
\section{Synthèse des travaux}
% Texte / Text.
Les résultats obtenus ont permis d’atteindre en tout ou en partie l’ensemble des sous-objectifs énoncés dans le chapitre~\ref{chapitre_methode}.%, voir Tableau~\ref{tab:synthese_travaux}.

\subsection{Projet Accorderie}
% TODO synthèse
La migration de la base de données a été réussie, mais elle est encore à ce jour en adaptation vers un modèle Odoo plus intégré au ERP. Les efforts ont été mis pour la création d’une interface utilisateur avec des technologies qui n’étaient pas à la base supportées dans Odoo.

\subsection{Projet Portail CEPPP}
% Synthèse
La signification que le nombre de lignes de XML aurait diminué, c’est que l’automate génère de base toutes les vues de tous les champs. Au moment de la ré-ingénierie, il y a eu beaucoup de nettoyage et de données XML effacées. Cependant, le développeur va mettre plus de code Python pour développer des logiques qui ne sont pas supportés par l’automate. Le Javascript ajouté sert à supporter les dates dans le portail. L’ajout de CSV sert pour l’ajout de permissions et rôles pour l’anonymisation.

Après la première migration par l’extraction du modèle de données par PHP, le client a pu testé la plateforme pour avoir une idée à quoi ressemblerait l’utilisation dans l’espace administration de leur modèle de données et ils ont fait des demandes de changement. Une analyse a été effectuée, nous avons utilisé le générateur de code pour générer les vues portails et fait une ré-ingénierie manuelle du modèle et des vues pour obtenir le résultat désiré. Une des fonctionnalités implémentés est l’anonymisation des données pour certains groupes d’utilisateurs, pour pouvoir visualiser des données sans avoir d’information personnelle sur le patient.




%%
%%  LIMITATIONS
%%
\section{Limitations de la solution proposée}\label{sec:Limitations}
% TODO dire en quoi ce que tu as fait améliore l’état de l’art décrit dans la littérature SYNTHÈSE
% TODO décrire les limitations/faiblesses de ce que tu as produit comme logiciel/résultats LIMITATION

\subsection{Couverture des tests}
% TODO limitation
Les tests devraient couvrir 100\% du code, cependant la couverture est de 84\% pour 3 raisons :

\begin{enumerate}
    \item Il y a du code fonctionnel non testé, il manque des tests;
    \item Il y a du code désuet qu’il faut nettoyer ou refactoriser;
    \item La gestion des erreurs n’est pas couverte, il faudrait les ignorer dans le test de couverture et faire des tests unitaires qui valide la gestion des erreurs.
\end{enumerate}


%%
%%  AMELIORATIONS FUTURES / FUTURE RESEARCH
%%
\section{Améliorations futures}
% Texte / Text.
% TODO dire ce qui doit être fait pour que les aspects d’utilisabilité de ton logiciel en terme de communauté, de libre, etc soient complets.

% TODO parler de la sympoïèse comme étant la suite sur la partie communautaire et la distribution de système


\subsection{Amélioration du générateur code}
% TODO amélioration future
Il faut intégrer la génération de code à l’intérieur des instances clientes dans l’objectif qu’ils soient accessibles de son gestionnaire de déploiement pour y ajouter les nouvelles fonctionnalités, démarrer la mise à jour, les tests, les améliorations, la migration, importation. Les instances clientes devraient proposer aux clients via les interfaces 

% Intégration de plusieurs types de réseaux de neurones accessibles librement, il doit être compatible avec le libre.

% Il faut suggérer aux clients des améliorations et de communiquer avec leurs gestionnaires de déploiements.

% Un générateur de code a été créé. L’embryon est créé, il faut terminer sa génération du générateur de code. Les déviances sont déjà créées

Une fois que le générateur de code aura atteint 100\% d’auto-génération, il restera limité à produire que les fonctionnalités qu’il utilise. Donc s’auto-générer fait office de test. Il faut faire des tests pour les fonctionnalités qu’il n’utilise pas (ou les combinaisons non utilisés) pour se reproduire. Il reste à auto-générer toutes ses techniques dans des modules qui font de l'héritage sur le générateur de code.

% Ajout de la génération de test de base, génération de documentation, mise à jour, migration lors d’une mise à jour sur les données, migration d’une mise à jour de la plateforme.

\subsubsection{Amélioration de l’architecture}
% TODO amélioration future
Parallélisation de tout le code tout le temps lorsque possible.

Automatisation de la configuration pour le déverminage, automatiser la détection des anomalies, améliorer l’interface no-code pour pouvoir accomplir les mêmes étapes que le mode de paramétrisation «Code hook». Supporter de nouvelles architectures dans la génération de code comme des applications Cordova pour le support mobile natif, des extensions Javascript dans Gnome Shell pour étendre les fonctionnalités du ERP directement sur le bureau d’un ordinateur sans passer par un navigateur web, générer des scripts de développement dans le projet ERPLibre qui ne dépendant pas d’Odoo, ou même supporter des applications embarqués.

De plus, il serait pertinent de supporter d’autres ERP externe comme NextERP ou Tryton qui sont des solutions libres. Cela va permettre la migration entre ERP des fonctionnalités et encourager l’utilisateur à prendre une solution entièrement AGPLv3 avec une communauté qui le supporte dans cette philosophie.

Problème d’extraction, il était dans le générateur de code au départ dans le développement, il y a donc une extraction à deux endroits qui rend complexe la gestion du code. Au fil de la progression du développement, l’extraction de données par rétro-ingénierie s’avère plus efficace que les méta-données du module dans le système Odoo.

L’auto-ingénierie sur la machine est en cours d’exécution sur le module de base, il faudrait aussi supporter les modules hérités qui sont définis comme des techniques.

Découper les fonctionnalités d'extraction et de génération, puis les séparer dans des modules des techniques du générateur de code.

Il faudrait réduire le nombre de technique dans Base pour qu’ils soient des modules externes par technique, ça faciliterait le changement d’architecture sur la gestion des méta-données, à adapter selon la rétro/ré-ingénierie.

% TODO mettre photo d'amélioration architecture

Les travaux d’amélioration devront être effectués après l’auto-génération.

\subsubsection{Amélioration de la gestion de projet et statistiques}
% TODO amélioration future
Le générateur de code doit offrir des outils de gestionnaire de projet pour suivre le développement, faire la liaison entre les demandes clients et les avancements des développeurs.

De plus, puisque l’état des méta-données évoluent, il devient difficile de faire le suivi des performances du générateur de code, puisqu’il vient aider dans les boucles d’itérations qui ne nécessitent pas de faire des commits, puisqu’on commit lorsque le tout est stable. Ainsi il faudrait faire des statistiques sur ces itérations pour évaluer la contribution du générateur.

Il manque l’analyse des différences de code sur les différentes sections générées.

\subsubsection{Suite du développement du générateur de code}
% TODO amélioration future
Il faudrait qu’il génère des tests fonctionnels, de la documentation fonctionnelle et développe la migration de données selon les changements des versions antérieurs. Un suivi des fonctionnalités selon les exigences clientes.

Une fois l’architecture mise à jour, la prochaine étape est de tester sa mise à niveau de tous les modules dans la communauté et détecter les techniques manquantes par supervision du développeur pour les implémenter. Une fois qu’il aura géré tous les modules de la communauté, on pourra implémenter la migration vers des mises à jour de la plateforme, c’est-à-dire vers Odoo 14, puis vers Odoo 16.

Une fois que l’auto-poïèse sera en place sur la gestion de la machine, la prochaine étape sera de faire l’auto-poïèse sur tout le code Odoo pour le développement de l’architecture.

\subsection{Projet Accorderie}
% TODO amélioration future
Maintenant qu’une plateforme sur le site web a été développée, il faudrait poursuivre la mise à jour du générateur de code pour supporter ce type de plateforme pour des projets futurs similaires.

\subsection{Projet Portail CEPPP}
% TODO amélioration future
Le générateur de code a été utilisé en début de projet et à fait économiser du temps de développement et réduit les erreurs possibles de retranscription du langage PHP au langage Python, ainsi que la génération des vues admin et portail. Cependant, l’automate a arrêté d’être utilisé au moment qu’on a commencé à diverger vers des fonctionnalités personnalisées qu’il ne pouvait pas supporter. Il faudra supporter ces fonctionnalités pour les futurs projets.

\subsection{NLP}
% TODO amélioration future
La technologie NLP va permettre de comprendre des textes rédigés par l’utilisateur et l’associer à des techniques de programmation.

% TODO Définir le besoin 
% TODO parler d’émotion et empathie

% Qu’est-ce qu’il faudrait développer pour se rendre au stade automate codeur?

Le NLP est une solution alternative pour interfacer avec l’utilisateur et communiquer avec pour développer des logiciels.

Suggestion d’explorer l’outil libre : Utiliser «HuggingFace» qui contient une grande communauté autour du développement d’un réseau de neurones pour faire du NLP par exemple, c’est compatible avec le logiciel libre.

\subsection{Support de développement de module dans la communauté Odoo}

ERPLibre supporte Odoo 12, puisque c’est lui qui supporte le plus de modules. Cependant, ces données représentent seulement ceux des repos utilisés par ERPLibre, il y a en beaucoup plus dans la communauté. C’est pourquoi il faut faire une recherche de ces modules dans la communauté et entreprises, et les rendre accessible par une base de données publiques~\footnote{Comme fait sur \url{https://odoo-code-search.com/}}.

Selon l’évolution, il faudrait migrer vers la version 14.0 en 2024. Donc il faut falloir supporter la migration de modules vers des versions supérieures.




%%
%%  CONCLUSION
%%

% Conclusion de la conclusion
% Nous avons besoin de continuer à faire de la recherche, pleins de question n'a pas été répondu, comment arriver à mettre en place une technopoïèse avec un automate générateur de code qui sait appliquer les 4 libertés? Quels seront les performances d'une machine autopoïèse? Va-t-on réussir à créer un robot codeur libre?

% L'autopoïèse qui sera développé servira à produire la représentation du contenu de ma thèse que je poursuis.

En conclusion, le robot logiciel libre codeur est en première phase de développement incluant la génération de code, l'interface avec l'utilisateur et la rétro-ingénierie pour appliquer de l'amélioration continue orienté au support d'un réseau d'entraide.
