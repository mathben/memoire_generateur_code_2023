%% Abstract
%%
%% Traduction anglaise fidèle et de qualité du résumé de la recherche écrit en français et non une traduction littérale. 
%%



\chapter*{ABSTRACT}\thispagestyle{headings}
\addcontentsline{toc}{compteur}{ABSTRACT}
%
\begin{otherlanguage}{english}
Enterprise resource planning (ERP) programmers develop the same functionality from one system to another with the same implementation technique from one feature to another. ERPs are complex and require long programming times, and error rates are high. Code writing automation is a solution for simplifying the programmer's work. A software robot developer, following the basics of industrialization, could be oriented towards the needs of the community and would allow to develop functionalities at an accelerated speed with the help of reverse engineering. The more available information is, the more efficient the robot will be, benefiting from all the advantages of free (as freedom) software i.e. use, copy, study and modify while distributing without restriction.

This dissertation presents and validates a concept of a software auto-reproducer using a reverse engineering technique in Python. The research focuses on the development of a self-developing technology enhanced by self-improvement with a self-engineering technique and also an auto-generator that is configurable to start a development chain on Odoo modules. To make this happen, we have developed several Odoo modules including code generation that allows to generate Odoo modules from metadata, apply reverse engineering to auto-reproduce an Odoo module to extract metadata, containing an interface that requires little or no code and other software practices to increase accessibility.

The free (as freedom) software robot coder is in the first phase of development including code generation, user interface and reverse engineering to apply continuous improvement oriented to support a peer support network. The machine is currently limited to web application generation on Odoo version 12.0 using ERPLibre 1.5.0.
% Written in English, the abstract is a brief summary similar to the previous
% section {\selectlanguage{french}(Résumé)}. However, this section is not a
% word for word translation of the abstract in French.

% The abstract is a brief statement of the subject matter, objectives, research questions or hypotheses, experimental methods and analysis of results. It also presents the main research conclusions and their possible applications. In general, an abstract should not exceed three pages.

% The abstract should provide an exact idea of the thesis or dissertation’s contents and it cannot be a simple enumeration of the manuscript’s parts. The goal is to precisely and concisely present the nature and scope of the research. An abstract should never include references or figures. If the thesis or the dissertation is in English, the résumé (French-language abstract) should come first followed by the abstract.

\end{otherlanguage}
