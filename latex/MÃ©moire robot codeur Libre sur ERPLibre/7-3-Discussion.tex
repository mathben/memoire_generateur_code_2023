\Chapter{DISCUSSION}\label{sec:Theme3}

\section{Interprétation des résultats}
Les résultats obtenus ont permis d’atteindre en tout ou en partie l’ensemble des sous-objectifs énoncés dans le chapitre~\ref{chapitre_methode}.%, voir Tableau~\ref{tab:synthese_travaux}.

\paragraph{SO-1 Accomplissements}
Simplification de l’écriture de technique dans le générateur de code avec l’utilisation de «f-string», de la bibliothèque «Code-writer» et de la bibliothèque «lxml».\\
Ajout de «Wizard» pour configurer le \texttt{MVC} selon des paramètres.\\
Amélioration de l’importation des bases de données externes.\\
Support de la génération de code par des données.\\
Augmentation de technique de génération de code, support des templates «Qweb», ajout du type de données géospatiale.

\paragraph{SO-1 Feuille de route}
Implémenter les fonctionnalités manquantes dans la GUI pour générer les MVC d’un module.\\
Augmenter le nombre de technologies à supporter l'importation des données.\\
Ajouter le support de génération sur différentes architectures.\\
Il manque des techniques de sécurité à personnaliser, comme l’anonymisation des données.\\
Générer automatiquement une documentation sur l’utilisation d’une technique. \\
Supporter la génération sur d’autres systèmes \texttt{ERP} libres tel que Tryton\footnote{\url{https://www.tryton.org}}.

\paragraph{SO-2 Accomplissements}
Extraction du code via l’utilisation d’un \texttt{AST} et extraction des méta-données dans les fichiers \texttt{XML}.\\
Amélioration continue sur la génération de code grâce à la reproduction à l’aide de l’extraction du code.\\
Un outil pour aider à la création de technique de génération à l’aide d’un générateur de générateur de code.\\
Le générateur de code est accompagné de tests de validation en reproduisant l’ensemble des techniques en démonstration.\\
La génération de code applique des règles de codage standardisées.

\paragraph{SO-2 Feuille de route}
Finaliser l’implémentation de l'auto génération sur l’automate.\\
Mise à jour des tests pour atteindre une couverture de code à 100\%.\\
Ajout de test sur les techniques d’extractions de code tel que le \texttt{PHP}.

\paragraph{SO-3 Accomplissements}
Ajout de nouvelles techniques et une classification de ceux-ci.\\
Rendre accessible une interface graphique pour paramétrer la génération de code.\\
Rendre accessible une interface de programmation pour utiliser toutes les fonctionnalités de l’automate.

\paragraph{SO-3 Feuille de route}
Ajout de paramètres pour faire plus de personnalisation sur les techniques et les séparer une par module.\\
Supporter les fonctionnalités manquantes sur toutes les techniques pour l’interface graphique.\\
Support l’accès à la création de méta-données par la rétro-ingénierie via l’interface graphique.

\paragraph{SO-4 Accomplissements}
Intégration dans un système de distribution via un Docker.

\paragraph{SO-4 Feuille de route}
Développer une synchronisation entre les instances pour permettre de la redondance.\\
Développer une gestion de son infrastructure via le générateur de code.\\
Faire participer l’automate à la maintenance de l’infrastructure de déploiement.\\
Utiliser d’autres systèmes de conteneur en distribution qui sont libres comme «Pod»~\footnote{\url{https://podman.io/}}.\\
L’automate doit avoir la capacité de valider techniquement si le logiciel est \texttt{AGPLv3} au moment de l’exécution

\paragraph{SO-5 Accomplissements}
Test de la génération sur un module existant de la communauté nommé «auto\_backup».\\
Des modules de gestion de projet ont été générés pour faire le suivi de la conception fonctionnelle et de l’amélioration continue.\\
Le projet Accorderie a bénéficié du générateur de code pour la migration de la base de données vers Odoo.\\
Le projet Portail CEPPP a bénéficié du générateur de code pour la migration du code \texttt{PHP} vers Odoo, ainsi que de l’aide au développement de la section Portail.

\paragraph{SO-5 Feuille de route}
L’automate doit supporter la demande de «Pull Request» sur les projets Git respectif lorsqu’il y a une amélioration. Il doit faire le suivi et s’assurer de suivre les règles de contributions de la communauté.\\
Développer d’autres modules de gestion de projet pour l’accompagnement dans le développement de projet client.\\
Développer des modules de gestion de communauté sur des projets logiciels libres.\\
Développer le suivi du développement des modules communautaires avec une traçabilité sur les résultats avec des métriques de génie logiciel.


% \begin{table}
% \caption{Synthèse des travaux}
% \centering
% \begin{tabular}{|p{0.6in}|p{2.7in}|p{2.7in}|}

% \hline
% \cellcolor[HTML]{d9d9d9}{\textbf{\shortstack[l]{Sous-\\objectif}}} & \cellcolor[HTML]{d9d9d9}{\textbf{Accomplissements}} & \cellcolor[HTML]{d9d9d9}{\textbf{Feuille de route}}\\\hline

% SO-1 & \shortstack[l]{Simplification de l’écriture de \\ technique dans le générateur de code \\ avec l’utilisation de «f-string», de la \\ bibliothèque «Code-writer» et de la \\ bibliothèque «lxml».\\
% Ajout de «Wizard» pour configurer \\ le \texttt{MVC} selon des paramètres.\\
% Amélioration de l’importation des \\ bases de données externes.\\
% Support de la génération de code par \\ des données.\\
% Augmentation de technique de \\ génération de code, support des \\ templates «Qweb», ajout du type de \\ données géospatiale.
% } & \shortstack[l]{Implémenter les fonctionnalités \\ manquantes dans la GUI pour \\ générer les MVC d’un module.\\
% Augmenter le nombre de \\ technologies à supporter l'importation \\ des données.\\
% Ajouter le support de génération \\ sur différentes architectures.\\
% Il manque des techniques de \\ sécurité à personnaliser, comme \\ l’anonymisation des données.\\
% Générer automatiquement une \\ documentation sur l’utilisation d’une \\ technique. \\
% Supporter la génération sur \\ d’autres systèmes \texttt{ERP} libres tel que \\ Tryton\footnote{\url{https://www.tryton.org}}.
% }\\\hline

% SO-2 & 
% \shortstack[l]{Extraction du code via l’utilisation \\ d’un \texttt{AST} et extraction des \\ méta-données dans les fichiers \texttt{XML}.\\
% Amélioration continue sur la \\ génération de code grâce à la \\ reproduction à l’aide de l’extraction \\ du code.\\
% Un outil pour aider à la création de \\ technique de génération à l’aide d’un \\ générateur de générateur de code.\\
% Le générateur de code est accompagné \\ de tests de validation en reproduisant \\ l’ensemble des techniques en \\ démonstration.\\
% La génération de code applique des \\ règles de codage standardisées.
% } &
% \shortstack[l]{Finaliser l’implémentation de \\l'auto-génération sur l’automate. \\
% Mise à jour des tests pour atteindre \\ une couverture de code à 100\%. \\
% Ajout de test sur les techniques \\ d’extractions de code tel que le \texttt{PHP}.
% } \\\hline

% \end{tabular}
% \label{tab:synthese_travaux}
% \end{table}

% \begin{table}[htbp]
% \centering
% \begin{tabularx}{\textwidth}{|X|X|X|}\hline
% \cellcolor[HTML]{d9d9d9}{\textbf{\shortstack[l]{Sous-\\objectif}}} & \cellcolor[HTML]{d9d9d9}{\textbf{Accomplissements}} & \cellcolor[HTML]{d9d9d9}{\textbf{Feuille de route}}\\\hline
% % \RaggedRight{A long-haired domestic cat, with a broad round head.}&
% % \RaggedRight{A short-legged, long-bodied, hound-type dog breed.}&
% % \RaggedRight{A breed of large dairy cattle originating in northern Holland.}\\\hline
% \end{tabularx}
% \end{table}

% \begin{table}
%     \begin{tabularx}{\textwidth}{|>{\hsize=0.5\hsize}X|X|X|}\hline
%         % Colonne 1 & Colonne 2 & Colonne 3 \\\hline
%         \cellcolor[HTML]{d9d9d9}{\textbf{\shortstack[l]{Sous-\\objectif}}} & \cellcolor[HTML]{d9d9d9}{\textbf{Accomplissements}} & \cellcolor[HTML]{d9d9d9}{\textbf{Feuille de route}}\\\hline
%         % Lorem ipsum dolor sit amet, consectetur adipiscing elit. & Pellentesque habitant morbi tristique senectus et netus et malesuada fames ac turpis egestas. & Pellentesque habitant morbi tristique senectus et netus et malesuada fames ac turpis egestas.  \\\hline
        
%         SO-1 & 
%         \shortstack[l]{Simplification de l’écriture de technique dans le générateur de code avec l’utilisation de «f-string», de la bibliothèque «Code-writer» et de la bibliothèque «lxml».\\
% Ajout de «Wizard» pour configurer \\ le \texttt{MVC} selon des paramètres.\\
% Amélioration de l’importation des \\ bases de données externes.\\
% Support de la génération de code par \\ des données.\\
% Augmentation de technique de \\ génération de code, support des \\ templates «Qweb», ajout du type de \\ données géospatiale.}
%  & \shortstack[l]{Implémenter les fonctionnalités \\ manquantes dans la GUI pour \\ générer les MVC d’un module.\\
% Augmenter le nombre de \\ technologies à supporter l'importation \\ des données.\\
% Ajouter le support de génération \\ sur différentes architectures.\\
% Il manque des techniques de \\ sécurité à personnaliser, comme \\ l’anonymisation des données.\\
% Générer automatiquement une \\ documentation sur l’utilisation d’une \\ technique. \\
% Supporter la génération sur \\ d’autres systèmes \texttt{ERP} libres tel que \\ Tryton\footnote{\url{https://www.tryton.org}}.}
        
%     \end{tabularx}
% \end{table}
