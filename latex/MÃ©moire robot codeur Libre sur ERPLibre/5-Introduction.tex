% Dans l'introduction, on présente le problème étudié et les buts
% poursuivis. L'introduction permet de faire connaître le cadre de la
% recherche et d'en préciser le domaine d'application. Elle fournit
% les précisions nécessaires en ce qui concerne le contexte de
% réalisation de la recherche, l'approche envisagée, l'évolution de
% la réalisation. En fait, l'introduction présente au lecteur ce
% qu'il doit savoir pour comprendre la recherche et en connaître la
% portée.
\Chapter{INTRODUCTION}\label{sec:Introduction}  % 10-12 lignes pour introduire le sujet.

%%
%%  Contexte et problématique
%%

\section{Contexte et problématique}

La valeur du marché des solutions \textit{ERP}, \textit{Enterprise Ressources Planning} ou progiciel de gestion des ressources d'une entreprise, s'établissait autour de 40 milliards USD mondialement en 2020 \cite{mordorintelligence_erp_2023,bigbang_erp_2023}. Le coût moyen par utilisateur, sur 5 ans, s'élevait à 9 000\$, pour une PME, en 2022 \cite{softwarepath_erp_2023}. Le développement de système \textit{ERP} est complexe, nécessite une maintenance exigeante et le risque d’introduire des erreurs est important~\cite{method_erp_system_2022}~\cite{wu_2006}.

La compagnie Odoo a réussi à devenir l'un des principaux fournisseurs d'\textit{ERP} à \textit{Open Source} au monde~\cite{ingenierie_system_information_hotel_odoo_2020}, avec une communauté active de développeurs et d'utilisateurs. Odoo relève plusieurs enjeux et défis dans les \textit{ERP}, notamment l'intégration de toutes les fonctions de l'entreprise, la personnalisation, la flexibilité, l'évolutivité et l'accessibilité.

Alors, comment serait-il possible d'accélérer le développement de fonctionnalités de la plateforme \textit{ERP} Odoo\footnote{Anciennement OpenERP, \textit{ERP} libre web, lien du projet \url{https://github.com/OCA/OCB}} 12 communautaire?

Pour répondre à ce besoin, la plateforme ERPLibre~\cite{ref_erplibre} a été créée dans l’objectif d’accélérer le développement de la plateforme Odoo communautaire. Ce mémoire va se concentrer sur la génération de code par des techniques de rétro-ingénierie et la gestion d’une communauté, dans un contexte d’un projet de logiciel libre, une solution développée, démontrée dans ce mémoire, en complément de la plateforme ERPLibre.

\section{Objectif et but}

Les objectifs des travaux effectués sont de développer un générateur de code pour accélérer le développement de fonctionnalité pour gérer les besoins d'une communauté, ainsi que de réaliser un auto-reproducteur, c'est-à-dire un générateur de code qui est capable de s'auto-générer, pour accélérer le développement de ce dernier.

Le but de ce mémoire est de montrer une solution qui contient des limitations, sur différents niveaux de production de développement logiciel, dans les domaines des \textit{ERP}, pour aider les réseaux d'entraide dans l'adaptation de leurs fonctionnalités tout en accélérant le développement avec des solutions libres.

% TODO problématique de devoir adapter les processus de l'entreprise aux capacité du ERP au lieu de donner un processus personnalisé

% TODO point de départ : «Code projet initial générateur de code» et donner REF

% Projet morceau de l'automate, son fondement libre, par sa création de technopoïèse.
% Il utiliser pour la liberté de l'automate d’offrir librement accès à ses fonctionnalités
% Il étudier comprendre et accepter son fonctionnement
% Il copier pour se l’approprier en respectant le copyright
% Il modifier son amélioration en tant que développement logiciel

% Combien ça prend de temps pour développer la migration pour autant de fonctionnalités? Ajouter des champs, migrer des champs en changeant leur nature.

\subsection{Choix de la plateforme \textit{ERP}}

Choisir une plateforme \textit{ERP} libre peut offrir des avantages significatifs en terme de coût, de flexibilité, de sécurité, de communauté et d'indépendance. Odoo a été retenu pour le projet puisqu'il répondait à ces critères, cependant, quelle est la version qui offre le plus de fonctionnalités?

\begin{table}
\caption{Tableau des dates de lancement du logiciel Odoo à partir de la version 6.0}
\centering
\begin{tabular}{|c|c|l|l|}

\hline
Légende & \cellcolor[HTML]{d9ead3}{\shortstack[l]{Version actuelle}} & \cellcolor[HTML]{fff2cc}{\shortstack[l]{Anciennes versions avec \\ maintenance étendue}} & \cellcolor[HTML]{f4cccc}{\shortstack[l]{Anciennes versions ou \\ fin de maintenance}}\\\hline

\hline\rowcolor[gray]{0.8}\color{black}
Odoo version & Date de lancement & \multicolumn{2}{|l|}{Commentaire}\\\hline
\cellcolor[HTML]{f4cccc}6.0/6.1 & octobre 2009 & \multicolumn{2}{|l|}{\shortstack[l]{Première publication sous AGPL, premier client \\ web}}\\\hline
\cellcolor[HTML]{f4cccc}7.0 & décembre 2012 & \multicolumn{2}{|l|}{} \\\hline
\cellcolor[HTML]{f4cccc}8.0 & septembre 2014 & \multicolumn{2}{|l|}{\shortstack[l]{Changement de nom pour Odoo, anciennement \\ OpenERP}}\\\hline
\cellcolor[HTML]{f4cccc}9.0 & novembre 2015 & \multicolumn{2}{|l|}{\shortstack[l]{Première publication des éditions «Community» \\ sous licence LGPLv3 et «Enterprise» sous licence \\ propriétaire.}}\\\hline
\cellcolor[HTML]{f4cccc}10.0 & octobre 2016 & \multicolumn{2}{|l|}{} \\\hline
\cellcolor[HTML]{f4cccc}11.0 & octobre 2017 & \multicolumn{2}{|l|}{} \\\hline
\cellcolor[HTML]{f4cccc}12.0 & octobre 2018 & \multicolumn{2}{|l|}{Version utilisée dans ERPLibre 1.5.0}\\\hline
\cellcolor[HTML]{fff2cc}13.0 & octobre 2019 & \multicolumn{2}{|l|}{}\\\hline
\cellcolor[HTML]{fff2cc}14.0 & octobre 2020 & \multicolumn{2}{|l|}{} \\\hline
\cellcolor[HTML]{fff2cc}15.0 & octobre 2021 & \multicolumn{2}{|l|}{} \\\hline
\cellcolor[HTML]{d9ead3}16.0 & octobre 2022 & \multicolumn{2}{|l|}{} \\\hline
\end{tabular}
\label{tab:choix_plateform_erp}
\end{table}

% Regarder l’évolution des modules dans OCA, en prenant leur nom de module, pour chaque version d’Odoo, selon une date déterminée, regarder le nombre de fois qu’il se répète. Ceci va indiquer la vitesse de migration des modules dans la communauté.

Odoo a été créé initialement sous le nom de TinyERP, en février 2005, cette plateforme a évolué au fil du temps, elle a été renommée pour OpenERP autour d'octobre 2009, passant sous licence AGPL.

À partir de janvier 2023, les versions 9.0 à 12.0 ne sont plus supportées officiellement par la compagnie Odoo, voir tableau~\ref{tab:choix_plateform_erp}, mais elles le sont encore par OCA. Ainsi, la version 16.0 est la version stable actuelle. La recherche de modules commence à partir de la version 9.0, là où débute la divergence entre une version communautaire et entreprise.

Au printemps 2020, Odoo version 12.0 a été choisi par ERPLibre\footnote{Première version de ERPLibre : \url{https://github.com/ERPLibre/ERPLibre/releases/tag/v0.1.0}.}. Une recherche de modules, par version d'Odoo, a été effectuée sur 11 Go de code et de données, sur le projet ERPLibre version 1.5.0, voir le tableau~\ref{tab:nb_module_version_odoo}. En date du premier janvier 2023, la version 12.0 est celle qui est retenue, avec 2 977 modules, puisqu'elle est celle qui a le plus de modules sur les 133 répertoires gérés par ERPLibre. Cette tendance pourrait changer dans le futur, mais la version 12.0 est celle dont fait l'objet ce mémoire.

Pour obtenir les résultats du tableau~\ref{tab:nb_module_version_odoo}, un script a été développé pour trouver la quantité de modules en cherchant dans les 133 répertoires Git\footnote{Logiciel de gestion de versions décentralisées}, puis dans toutes les versions d'Odoo.

Parfois, la quantité de modules diminue d'une année à l'autre. Il y a création d'une nouvelle branche, lors d'une nouvelle version, qui est la suite de la version précédente. Par exemple, dans le tableau~\ref{tab:nb_module_version_odoo}, la version 10.0, entre 2017 et 2018, il y a une réduction de 171 modules dans les répertoires d'entreprise, mais il y a eu seulement 4 mois pour faire le nettoyage. Les méthodes de mises à jour ont évolué depuis.

De plus, les chiffres du tableau~\ref{tab:nb_module_version_odoo} semblent démontrer que les versions paires d'Odoo sont plus populaires que les versions impaires. Cependant, la communauté d'Odoo est bien plus grosse que ce que les 133 répertoires semblent suggérer.

Dans la section «total» du tableau~\ref{tab:nb_module_version_odoo}, la section unique signifie que la somme va ignorer les doublons. En date du premier janvier 2023, il y a eu, au total, 17 309 modules, dont 6 063 modules uniques. Cela signifie qu'il y a 11 246 modules en doublon. Hors, le code diffère d'une version à l'autre, même si c'est un doublon, il peut avoir des bogues ou des fonctionnalités différentes entre elles.

% Prochain tableau
\begin{table}
\caption{Nombre de modules, contenant un manifest installable, par version Odoo, à partir du premier janvier, minuit, par année, sur la plateforme ERPLibre 1.5.0.}
\centering
\begin{tabular}{|l|l|l|l|l|l|l|l|l|}
\hline

Légende & \multicolumn{2}{|l|}{\shortstack[l]{Total\\\textcolor[HTML]{274e13}{OCA}\\\textcolor[HTML]{4c1130}{Entreprise}}} & \multicolumn{2}{|c|}{\cellcolor[HTML]{d9ead3}{\shortstack[l]{Version \\ actuelle}}} & \multicolumn{2}{|c|}{\cellcolor[HTML]{fff2cc}{\shortstack[l]{Anciennes \\ versions avec \\ maintenance \\ étendue}}} & \multicolumn{2}{|c|}{\cellcolor[HTML]{f4cccc}{\shortstack[l]{Anciennes \\ versions ou \\ fin de \\ maintenance}}}\\\hline

\multicolumn{9}{|l|}{\shortstack[l]{17309/\textcolor[HTML]{274e13}{10728}/\textcolor[HTML]{4c1130}{6581} modules à supporter le 1er janvier 2023 \\ 17465/\textcolor[HTML]{274e13}{10952}/\textcolor[HTML]{4c1130}{6513} modules le 15 février 2023 \\ 156/\textcolor[HTML]{274e13}{132}/\textcolor[HTML]{4c1130}{24} nouveaux modules en 31 jours, durant janvier 2023}}\\\hline

Odoo version & 2016 & 2017 & 2018 & 2019 & 2020 & 2021 & 2022 & 2023 \\\hline

6.1 &
\cellcolor[HTML]{fff2cc}{\shortstack[r]{295 \\ \textcolor[HTML]{274e13}{269} \\ \textcolor[HTML]{4c1130}{26}}} & 
\cellcolor[HTML]{f4cccc}{\shortstack[r]{299 \\ \textcolor[HTML]{274e13}{270} \\ \textcolor[HTML]{4c1130}{29}}} &
\cellcolor[HTML]{f4cccc}{\shortstack[r]{299 \\ \textcolor[HTML]{274e13}{270} \\ \textcolor[HTML]{4c1130}{29}}} &
\cellcolor[HTML]{f4cccc}{\shortstack[r]{299 \\ \textcolor[HTML]{274e13}{270} \\ \textcolor[HTML]{4c1130}{36}}} &
\cellcolor[HTML]{f4cccc}{\shortstack[r]{299 \\ \textcolor[HTML]{274e13}{270} \\ \textcolor[HTML]{4c1130}{36}}} &
\cellcolor[HTML]{f4cccc}{\shortstack[r]{299 \\ \textcolor[HTML]{274e13}{270} \\ \textcolor[HTML]{4c1130}{36}}} &
\cellcolor[HTML]{f4cccc}{\shortstack[r]{299 \\ \textcolor[HTML]{274e13}{270} \\ \textcolor[HTML]{4c1130}{36}}} &
\cellcolor[HTML]{f4cccc}{\shortstack[r]{299 \\ \textcolor[HTML]{274e13}{270} \\ \textcolor[HTML]{4c1130}{36}}} \\\hline

7.0 &
\cellcolor[HTML]{fff2cc}{\shortstack[r]{637 \\ \textcolor[HTML]{274e13}{597} \\ \textcolor[HTML]{4c1130}{40}}} & 
\cellcolor[HTML]{fff2cc}{\shortstack[r]{633 \\ \textcolor[HTML]{274e13}{592} \\ \textcolor[HTML]{4c1130}{41}}} &
\cellcolor[HTML]{f4cccc}{\shortstack[r]{634 \\ \textcolor[HTML]{274e13}{593} \\ \textcolor[HTML]{4c1130}{41}}} &
\cellcolor[HTML]{f4cccc}{\shortstack[r]{635 \\ \textcolor[HTML]{274e13}{594} \\ \textcolor[HTML]{4c1130}{41}}} &
\cellcolor[HTML]{f4cccc}{\shortstack[r]{669 \\ \textcolor[HTML]{274e13}{619} \\ \textcolor[HTML]{4c1130}{50}}} &
\cellcolor[HTML]{f4cccc}{\shortstack[r]{669 \\ \textcolor[HTML]{274e13}{619} \\ \textcolor[HTML]{4c1130}{50}}} &
\cellcolor[HTML]{f4cccc}{\shortstack[r]{669 \\ \textcolor[HTML]{274e13}{619} \\ \textcolor[HTML]{4c1130}{50}}} &
\cellcolor[HTML]{f4cccc}{\shortstack[r]{671 \\ \textcolor[HTML]{274e13}{619} \\ \textcolor[HTML]{4c1130}{52}}} \\\hline

8.0 &
\cellcolor[HTML]{fff2cc}{\shortstack[r]{741 \\ \textcolor[HTML]{274e13}{597} \\ \textcolor[HTML]{4c1130}{144}}} & 
\cellcolor[HTML]{fff2cc}{\shortstack[r]{1092 \\ \textcolor[HTML]{274e13}{907} \\ \textcolor[HTML]{4c1130}{185}}} &
\cellcolor[HTML]{fff2cc}{\shortstack[r]{1215 \\ \textcolor[HTML]{274e13}{996} \\ \textcolor[HTML]{4c1130}{219}}} &
\cellcolor[HTML]{f4cccc}{\shortstack[r]{1265 \\ \textcolor[HTML]{274e13}{1027} \\ \textcolor[HTML]{4c1130}{238}}} &
\cellcolor[HTML]{f4cccc}{\shortstack[r]{1290 \\ \textcolor[HTML]{274e13}{1036} \\ \textcolor[HTML]{4c1130}{254}}} &
\cellcolor[HTML]{f4cccc}{\shortstack[r]{1297 \\ \textcolor[HTML]{274e13}{1043} \\ \textcolor[HTML]{4c1130}{254}}} &
\cellcolor[HTML]{f4cccc}{\shortstack[r]{1340 \\ \textcolor[HTML]{274e13}{1049} \\ \textcolor[HTML]{4c1130}{291}}} &
\cellcolor[HTML]{f4cccc}{\shortstack[r]{1341 \\ \textcolor[HTML]{274e13}{1050} \\ \textcolor[HTML]{4c1130}{291}}} \\\hline

9.0 &
\cellcolor[HTML]{d9ead3}{\shortstack[r]{135 \\ \textcolor[HTML]{274e13}{46} \\ \textcolor[HTML]{4c1130}{89}}} & 
\cellcolor[HTML]{fff2cc}{\shortstack[r]{456 \\ \textcolor[HTML]{274e13}{346} \\ \textcolor[HTML]{4c1130}{110}}} &
\cellcolor[HTML]{fff2cc}{\shortstack[r]{725 \\ \textcolor[HTML]{274e13}{602} \\ \textcolor[HTML]{4c1130}{123}}} &
\cellcolor[HTML]{fff2cc}{\shortstack[r]{776 \\ \textcolor[HTML]{274e13}{643} \\ \textcolor[HTML]{4c1130}{133}}} &
\cellcolor[HTML]{f4cccc}{\shortstack[r]{796 \\ \textcolor[HTML]{274e13}{659} \\ \textcolor[HTML]{4c1130}{137}}} &
\cellcolor[HTML]{f4cccc}{\shortstack[r]{803 \\ \textcolor[HTML]{274e13}{666} \\ \textcolor[HTML]{4c1130}{137}}} &
\cellcolor[HTML]{f4cccc}{\shortstack[r]{844 \\ \textcolor[HTML]{274e13}{666} \\ \textcolor[HTML]{4c1130}{178}}} &
\cellcolor[HTML]{f4cccc}{\shortstack[r]{850 \\ \textcolor[HTML]{274e13}{669} \\ \textcolor[HTML]{4c1130}{181}}} \\\hline

10.0 &
\cellcolor[HTML]{cccccc}{} & 
\cellcolor[HTML]{d9ead3}{\shortstack[r]{523 \\ \textcolor[HTML]{274e13}{111} \\ \textcolor[HTML]{4c1130}{412}}} &
\cellcolor[HTML]{fff2cc}{\shortstack[r]{954 \\ \textcolor[HTML]{274e13}{713} \\ \textcolor[HTML]{4c1130}{241}}} &
\cellcolor[HTML]{fff2cc}{\shortstack[r]{1 537 \\ \textcolor[HTML]{274e13}{953} \\ \textcolor[HTML]{4c1130}{584}}} &
\cellcolor[HTML]{fff2cc}{\shortstack[r]{1 647 \\ \textcolor[HTML]{274e13}{1 047} \\ \textcolor[HTML]{4c1130}{600}}} &
\cellcolor[HTML]{f4cccc}{\shortstack[r]{1 685 \\ \textcolor[HTML]{274e13}{1 085} \\ \textcolor[HTML]{4c1130}{600}}} &
\cellcolor[HTML]{f4cccc}{\shortstack[r]{1 754 \\ \textcolor[HTML]{274e13}{1 109} \\ \textcolor[HTML]{4c1130}{645}}} &
\cellcolor[HTML]{f4cccc}{\shortstack[r]{1 765 \\ \textcolor[HTML]{274e13}{1 120} \\ \textcolor[HTML]{4c1130}{645}}} \\\hline

11.0 &
\cellcolor[HTML]{cccccc}{} & 
\cellcolor[HTML]{cccccc}{} &
\cellcolor[HTML]{d9ead3}{\shortstack[r]{288 \\ \textcolor[HTML]{274e13}{77} \\ \textcolor[HTML]{4c1130}{211}}} &
\cellcolor[HTML]{fff2cc}{\shortstack[r]{1 398 \\ \textcolor[HTML]{274e13}{658} \\ \textcolor[HTML]{4c1130}{740}}} &
\cellcolor[HTML]{fff2cc}{\shortstack[r]{1 710 \\ \textcolor[HTML]{274e13}{929} \\ \textcolor[HTML]{4c1130}{781}}} &
\cellcolor[HTML]{fff2cc}{\shortstack[r]{1 797 \\ \textcolor[HTML]{274e13}{1 000} \\ \textcolor[HTML]{4c1130}{797}}} &
\cellcolor[HTML]{f4cccc}{\shortstack[r]{1 860 \\ \textcolor[HTML]{274e13}{1 023} \\ \textcolor[HTML]{4c1130}{837}}} &
\cellcolor[HTML]{f4cccc}{\shortstack[r]{1 869 \\ \textcolor[HTML]{274e13}{1 032} \\ \textcolor[HTML]{4c1130}{864}}} \\

\noalign{\hrule height 2pt}
\multicolumn{1}{!{\vrule width 2pt}l!{\vrule width 1pt}}{\textbf{12.0}} &
\cellcolor[HTML]{cccccc}{} & 
\cellcolor[HTML]{cccccc}{} &
\cellcolor[HTML]{cccccc}{} &
\multicolumn{1}{!{\vrule width 2pt}r!{\vrule width 1pt}}{\textbf{\cellcolor[HTML]{d9ead3}{\shortstack[r]{784 \\ \textcolor[HTML]{274e13}{137} \\ \textcolor[HTML]{4c1130}{647}}}}} &
\multicolumn{1}{!{\vrule width 2pt}r!{\vrule width 1pt}}{\textbf{\cellcolor[HTML]{fff2cc}{\shortstack[r]{1 837 \\ \textcolor[HTML]{274e13}{993} \\ \textcolor[HTML]{4c1130}{844}}}}} &
\multicolumn{1}{!{\vrule width 2pt}r!{\vrule width 1pt}}{\textbf{\cellcolor[HTML]{fff2cc}{\shortstack[r]{2 503 \\ \textcolor[HTML]{274e13}{1 464} \\ \textcolor[HTML]{4c1130}{1 039}}}}} &
\multicolumn{1}{!{\vrule width 2pt}r!{\vrule width 1pt}}{\textbf{\cellcolor[HTML]{fff2cc}{\shortstack[r]{2 851 \\ \textcolor[HTML]{274e13}{1 633} \\ \textcolor[HTML]{4c1130}{1 218}}}}} &
\multicolumn{1}{!{\vrule width 2pt}r!{\vrule width 1pt}}{\textbf{\cellcolor[HTML]{f4cccc}{\shortstack[r]{2 977 \\ \textcolor[HTML]{274e13}{1 693} \\ \textcolor[HTML]{4c1130}{1 284}}}}} \\
\noalign{\hrule height 2pt}

13.0 &
\cellcolor[HTML]{cccccc}{} & 
\cellcolor[HTML]{cccccc}{} &
\cellcolor[HTML]{cccccc}{} &
\cellcolor[HTML]{cccccc}{} &
\cellcolor[HTML]{d9ead3}{\shortstack[r]{617 \\ \textcolor[HTML]{274e13}{115} \\ \textcolor[HTML]{4c1130}{502}}} &
\cellcolor[HTML]{fff2cc}{\shortstack[r]{1 445 \\ \textcolor[HTML]{274e13}{844} \\ \textcolor[HTML]{4c1130}{601}}} &
\cellcolor[HTML]{fff2cc}{\shortstack[r]{2 024 \\ \textcolor[HTML]{274e13}{1 310} \\ \textcolor[HTML]{4c1130}{714}}} &
\cellcolor[HTML]{fff2cc}{\shortstack[r]{2 241 \\ \textcolor[HTML]{274e13}{1 506} \\ \textcolor[HTML]{4c1130}{735}}} \\\hline

14.0 &
\cellcolor[HTML]{cccccc}{} & 
\cellcolor[HTML]{cccccc}{} &
\cellcolor[HTML]{cccccc}{} &
\cellcolor[HTML]{cccccc}{} &
\cellcolor[HTML]{cccccc}{} &
\cellcolor[HTML]{d9ead3}{\shortstack[r]{906 \\ \textcolor[HTML]{274e13}{129} \\ \textcolor[HTML]{4c1130}{777}}} &
\cellcolor[HTML]{fff2cc}{\shortstack[r]{2 150 \\ \textcolor[HTML]{274e13}{1 143} \\ \textcolor[HTML]{4c1130}{1 007}}} &
\cellcolor[HTML]{fff2cc}{\shortstack[r]{2 648 \\ \textcolor[HTML]{274e13}{1 698} \\ \textcolor[HTML]{4c1130}{950}}} \\\hline

15.0 &
\cellcolor[HTML]{cccccc}{} & 
\cellcolor[HTML]{cccccc}{} &
\cellcolor[HTML]{cccccc}{} &
\cellcolor[HTML]{cccccc}{} &
\cellcolor[HTML]{cccccc}{} &
\cellcolor[HTML]{cccccc}{} &
\cellcolor[HTML]{d9ead3}{\shortstack[r]{786 \\ \textcolor[HTML]{274e13}{96} \\ \textcolor[HTML]{4c1130}{690}}} &
\cellcolor[HTML]{fff2cc}{\shortstack[r]{1 669 \\ \textcolor[HTML]{274e13}{865} \\ \textcolor[HTML]{4c1130}{804}}} \\\hline

16.0 &
\cellcolor[HTML]{cccccc}{} & 
\cellcolor[HTML]{cccccc}{} &
\cellcolor[HTML]{cccccc}{} &
\cellcolor[HTML]{cccccc}{} &
\cellcolor[HTML]{cccccc}{} &
\cellcolor[HTML]{cccccc}{} &
\cellcolor[HTML]{cccccc}{} &
\cellcolor[HTML]{d9ead3}{\shortstack[l]{972 \\ \textcolor[HTML]{274e13}{206} \\ \textcolor[HTML]{4c1130}{766}}} \\\hline

\multicolumn{9}{|c|}{Total}\\\hline

Somme &
\shortstack[r]{1 808 \\ \textcolor[HTML]{274e13}{1 509} \\ \textcolor[HTML]{4c1130}{299}} & 
\shortstack[r]{3 003 \\ \textcolor[HTML]{274e13}{2 226} \\ \textcolor[HTML]{4c1130}{777}} &
\shortstack[r]{4 115 \\ \textcolor[HTML]{274e13}{3 251} \\ \textcolor[HTML]{4c1130}{864}} &
\shortstack[r]{6 701 \\ \textcolor[HTML]{274e13}{4 282} \\ \textcolor[HTML]{4c1130}{2 419}} &
\shortstack[r]{8 872 \\ \textcolor[HTML]{274e13}{5 668} \\ \textcolor[HTML]{4c1130}{3 204}} &
\shortstack[r]{11 411 \\ \textcolor[HTML]{274e13}{7 120} \\ \textcolor[HTML]{4c1130}{4 291}} &
\shortstack[r]{14 584 \\ \textcolor[HTML]{274e13}{8 918} \\ \textcolor[HTML]{4c1130}{5 666}} &
\shortstack[r]{17 309 \\ \textcolor[HTML]{274e13}{10 728} \\ \textcolor[HTML]{4c1130}{6 581}} \\\hline

Support Odoo &
\shortstack[r]{1 808 \\ \textcolor[HTML]{274e13}{1 509} \\ \textcolor[HTML]{4c1130}{299}} & 
\shortstack[r]{2 704 \\ \textcolor[HTML]{274e13}{1 956} \\ \textcolor[HTML]{4c1130}{748}} &
\shortstack[r]{3 182 \\ \textcolor[HTML]{274e13}{2 388} \\ \textcolor[HTML]{4c1130}{794}} &
\shortstack[r]{4 495 \\ \textcolor[HTML]{274e13}{2 391} \\ \textcolor[HTML]{4c1130}{2 104}} &
\shortstack[r]{5 811 \\ \textcolor[HTML]{274e13}{3 084} \\ \textcolor[HTML]{4c1130}{2 727}} &
\shortstack[r]{6 651 \\ \textcolor[HTML]{274e13}{3 437} \\ \textcolor[HTML]{4c1130}{3 214}} &
\shortstack[r]{7 811 \\ \textcolor[HTML]{274e13}{4 182} \\ \textcolor[HTML]{4c1130}{3 629}} &
\shortstack[r]{7 530 \\ \textcolor[HTML]{274e13}{4 275} \\ \textcolor[HTML]{4c1130}{3 255}} \\\hline

Unique &
\shortstack[r]{1 244 \\ \textcolor[HTML]{274e13}{1 047} \\ \textcolor[HTML]{4c1130}{197}} & 
\shortstack[r]{1 995 \\ \textcolor[HTML]{274e13}{1 435} \\ \textcolor[HTML]{4c1130}{560}} &
\shortstack[r]{2 260 \\ \textcolor[HTML]{274e13}{1 845} \\ \textcolor[HTML]{4c1130}{415}} &
\shortstack[r]{3 214 \\ \textcolor[HTML]{274e13}{2 172} \\ \textcolor[HTML]{4c1130}{1 042}} &
\shortstack[r]{3 927 \\ \textcolor[HTML]{274e13}{2 610} \\ \textcolor[HTML]{4c1130}{1 317}} &
\shortstack[r]{4 676 \\ \textcolor[HTML]{274e13}{3 080} \\ \textcolor[HTML]{4c1130}{1 596}} &
\shortstack[r]{5 452 \\ \textcolor[HTML]{274e13}{3 572} \\ \textcolor[HTML]{4c1130}{1 880}} &
\shortstack[r]{6 063 \\ \textcolor[HTML]{274e13}{3 980} \\ \textcolor[HTML]{4c1130}{2 083}} \\\hline

\multicolumn{9}{|l|}{\shortstack[l]{133/\textcolor[HTML]{274e13}{72}/\textcolor[HTML]{4c1130}{61} répertoires de modules dans ERPLibre 1.5.0}}\\\hline

\end{tabular}
\label{tab:nb_module_version_odoo}
\end{table}

\newpage

\subsection{Introduction Accorderie}

La première étude de cas du présent mémoire a été effectuée sur l'Accorderie. Ce réseau d'entraide québécois était à la recherche d'une plateforme améliorée, avec des technologies plus récentes. Les membres ont commencé, depuis quelques années, à utiliser des plateformes alternatives, à cause de l'émergence des plateformes de réseaux sociaux, pour communiquer et échanger des services, sans passer par la plateforme Espace Membre. En ajoutant des fonctionnalités, dont l'automatisation\footnote{La gestion des feuilles de temps est manuelle avec validation par un membre de la communauté.} des processus d'échange de temps, la plateforme devient plus attractive.

% L’objectif du projet en collaboration avec l’Accorderie est de faire une plateforme améliorée avec des technologies plus récentes pour contrer l’effet des réseaux sociaux qui est devenu un intermédiaire intéressant pour échanger entre les membres, ainsi que d’automatiser les processus d’échange de temps, qui demande actuellement une gestion manuelle.

% WHY samuel voulait faire une plateforme d'échange de temps et j'étais d'accord puisque c'est une manière pour promouvoir le libre auprès des communautés via des moyens technologiques et leur rendre accessible l'automate codeur pour les aider avec leur entreprise et même aider dans les mouvements en transition.

% Nous eux besoin de répondre à mettre en place un réseau d'entraide basé sur des concepts d'échange de temps. \footnote{citation de samuel à effectuer} pour répondre à des besoins qui ne peuvent pas être fait par les entreprises.
% TODO besoin de faire 

% Nous nous sommes rencontrés, avec Nadia Mohammed-Azizi, directrice du Réseau Accorderie, pour comprendre les problématiques et nous avons par la suite fait une analyse fonctionnelle.

% L’objectif du projet en collaboration avec l’Accorderie est de faire une plateforme améliorée avec des technologies plus récentes pour contrer l’effet des réseaux sociaux qui est devenu un intermédiaire intéressant pour échanger entre les membres, ainsi que d’automatiser les processus d’échange de temps, qui demande actuellement une gestion manuelle.

Nous avons décelé que le présent projet pourrait répondre aux besoins de l'Accorderie, avec le générateur de modules Odoo, qui permet de faciliter, entre autres, la maintenance dans le temps. De plus, la plateforme ERPLibre a le potentiel de leur éviter des coûts sur les licences, du développement redondant et leur donner des fonctionnalités personnalisées. Ainsi, nous avons obtenu accès au code source PHP de la plateforme Espace Membre dont le droit d'auteur mentionne l’année 2007, par la compagnie GRF Ressource Informatique. De plus, nous avons aussi eu accès à la base de données et, selon les archives, le premier échange tracé est le premier janvier 2003. «Le 3 juin 2002, l’Accorderie de Québec est officiellement constituée en tant qu’organisme à but non lucratif. Sa mission : lutter contre la pauvreté et l’exclusion sociale, ainsi que favoriser la mixité sociale»\cite{erudit_accorderie_2014}. La plateforme aurait eu plusieurs mises à jour au fil du temps, cela nous donnait un cas réel empirique, avec des données et des utilisateurs réels, pour rendre concret une plateforme d'échange de temps. 

\subsection{Introduction CEPPP}

La seconde étude de cas porte sur le CEPPP, Centre d’excellence sur le partenariat avec les patients et le public qui était à la recherche d'une plateforme pour la gestion du partenariat avec les patients et le public. Le Portail des partenaires (“Portail”) du CEPPP est issu de la fusion de communautés de patients partenaires, entre autres, de la Direction collaboration et partenariat patient (DCPP) de la Faculté de médecine de l’Université de Montréal, et celle du Centre de recherche du Centre Hospitalier de l’Université de Montréal (CR-CHUM). Le Portail est un outil qui vient soutenir les activités de recrutement et de recherche sur les pratiques de partenariat. Donc, une solution~\cite{github_ceppp_crm}, était déjà présente, mais elle était incomplète. Il manquait, dans cette solution, la gestion de l'anonymisation et l'interface demandait beaucoup de navigation pour atteindre l'information et exigeait un clic de souris pour chaque champs par section.

Le présent projet venait pallier aux problèmes rencontrés en facilitant et en accélérant le développement, en permettant une personnalisation et en développant l'anonymisation des données et était donc tout indiqué comme second cas pratique d'étude.

% À l'instar de l'Accorderie, cela permettait aussi un cas réel d'utilisation empirique du projet afin d'améliorer et de comprendre ce dont le générateur de code a besoin. 

% Selon le développement initial \footnote{Lien du projet Git \url{https://github.com/lerenardprudent/ceppp_crm/tree/master}}, le développement a été commencé le 25 mars 2019 et a terminé le 27 décembre 2019, utilisant la plateforme SuiteCRM en PHP, sous licence AGPLv3. Le codage a été fait directement sur la plateforme, rendant plus difficile la mise à jour, puisqu’il faut éviter les conflits de code.

\section{Organisation générale du document}

Le chapitre \ref{sec:RevLitt} présente une revue de littérature concernant les termes de robot logiciel développeur, la génération de code, les logiciels libres et \textit{Open 
Source}, la sécurité en lien avec le développement logiciel, la complexité du développement des ERP, le \textit{DevOps}, les interfaces logicielles \textit{no-code / low-code}, la création de communauté et la poïèse. Il termine par le cadre conceptuel, qui contient des explications sur un générateur de code accompagné de rétro-ingénierie, ainsi que des exemples de générateur de code avec une méthode de test. Dans le chapitre \ref{sec:Theme1}, nous verrons en détail la méthodologie de travail et les objectifs de développement pour réaliser le générateur de code qui est séparé en 5 points : générateur de code, application de la rétro-ingénierie, l'interface du générateur de code, le déploiement du générateur de code et application dans des réseaux d'entraide. Le chapitre \ref{sec:Theme2} comporte les résultats présentés, ainsi qu'une discussion générale. Enfin, le chapitre \ref{sec:Conclusion} conclut ce mémoire avec une synthèse critique et des pistes pour des recherches futures.