% Résumé du mémoire.
% Abstract in French.
%
\chapter*{RÉSUMÉ}\thispagestyle{headings}
\addcontentsline{toc}{compteur}{RÉSUMÉ}

% Le résumé est un bref exposé du sujet traité, des objectifs visés, des hypothèses émises, des méthodes expérimentales utilisées et de l'analyse des résultats obtenus. On y présente également les principales conclusions de la recherche ainsi que ses applications éventuelles. En général, un résumé ne dépasse pas trois pages.

% Le résumé doit donner une idée exacte du contenu du mémoire ou de la thèse. Ce ne peut pas être une simple énumération des parties du manuscrit. Le but est de présenter de façon précise et concise la nature, l’envergure de la recherche, les sujets traités, les questions de recherche ou les hypothèses soulevées, les méthodes utilisées, les principaux résultats ainsi que les conclusions retenues. Un résumé ne doit jamais comporter de références ou de figures. 

