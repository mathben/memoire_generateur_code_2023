% Résumé du mémoire.
% Abstract in French.
%
\chapter*{RÉSUMÉ}\thispagestyle{headings}
\addcontentsline{toc}{compteur}{RÉSUMÉ}

% Introduction
Les programmeurs de progiciels de planification des ressources d’entreprise (ERP) développent les mêmes fonctionnalités d’un système à l’autre avec la même technique d’implémentation d’une fonctionnalité à une autre. Les ERP sont complexes et demandent de longues durées de programmation, les taux d’erreurs sont élevés. L’automatisation d’écriture de code est une solution pour la simplification du travail du programmeur. Un robot logiciel développeur, suivant les bases de l’industrialisation, pourrait être orienté vers les besoins de la communauté et permettrait de développer des fonctionnalités à une vitesse accélérée à l’aide de la rétro-ingénierie. Plus la quantité d’information est disponible, plus le robot sera efficace, tirant tous les avantages du logiciel libre i.e. utiliser, copier, étudier et modifier tout en distribuant sans restriction.

% action


% vision
% La poïèse de la production automatisé sous forme de procédure et de technologie. Pour un robot, la technopoïèse est un système pour fabriquer des technologies pour supporter les actions humanitaires. La robotique est. Son interface doit être de type avec peu ou pas de code (LCNC). Besoin de NLP pour comprendre la communication humaine.

% Présentation en suggestion\footnote{Les guides peuvent être adapté aux nouveaux contextes d'état d'urgence} de guide pour gestionnaire de projet qui doivent être acquise.

Ce mémoire va démontrer un concept d'un auto-reproducteur logiciel utilisant une technique de rétro-ingénierie en Python. Notre recherche va porter sur le développement d'une technologie auto-développeur bonifiée par de l'auto-amélioration avec une technique d'auto-ingénierie et aussi un auto-générateur qui est paramétrable pour démarrer une chaîne de développement sur des modules Odoo. Pour y arriver, nous avons développé plusieurs modules Odoo incluant la génération de code qui permet de générer des modules Odoo à partir de méta-données, d'appliquer de la rétro-ingénierie pour faire de l'auto-reproduction sur un module Odoo pour extraire les méta-données, contenant une interface qui nécessite peu ou pas de code et d'autres pratiques logicielles pour augmenter l'accessibilité. Nous allons faire des liaisons entre la gestion d'une communauté autour d'un projet technologique libre et le démarrage pour un gestionnaire d'un réseau d'entraide, assisté par un générateur de code automatisé pour mettre en place de l'amélioration continue sur le développement et les habitudes des participants.

La machine est présentement limitée à la génération d'application web sur Odoo version 12.0 en utilisant ERPLibre 1.5.0. L'auto-poïèse est sur le point d'être terminée, l'allo-poïèse fonctionne bien. Les travaux ne sont pas encore terminés et ce mémoire présente l'état d'avancement des résultats.


% Le résumé est un bref exposé du sujet traité, des objectifs visés, des hypothèses émises, des méthodes expérimentales utilisées et de l'analyse des résultats obtenus. On y présente également les principales conclusions de la recherche ainsi que ses applications éventuelles. En général, un résumé ne dépasse pas trois pages.

% Le résumé doit donner une idée exacte du contenu du mémoire ou de la thèse. Ce ne peut pas être une simple énumération des parties du manuscrit. Le but est de présenter de façon précise et concise la nature, l’envergure de la recherche, les sujets traités, les questions de recherche ou les hypothèses soulevées, les méthodes utilisées, les principaux résultats ainsi que les conclusions retenues. Un résumé ne doit jamais comporter de références ou de figures. 

